\documentclass[../main.tex]{subfiles}

\begin{document}
\setcounter{chapter}{4}
\chapter{Fonctions usuelles}
\tableofcontents
\clearpage

\setcounter{section}{1}
\section{Propriétés du logarithme}
Par définition, $\ln$ est définie et dérvable sur $\mathbb{R}_+^*$ et :
\begin{align*}
    \forall x > 0, \ln'(x) = \frac{1}{x}
\end{align*}

\noindent On montre par récurrence sur $n \geq 1$ que 
\begin{align*}
    \text{"$\ln$ est dérivable $n$ fois et } \forall n > 0, \ln^{(n)}(x) = \frac{(-1)^{n - 1}(n - 1)!}{x^n} \text{"}
\end{align*}
\underline{Initialisation :} \\
La propriété est vraie pour $n = 1$. \\ \\

\noindent \underline{Hérédité :} \\
Si elle est vraie pour $n \geq 1$, par théorème d'opérations, $\ln^{(n)}$ est encore dérivable et :
\begin{align*}
    \forall x > 0, ln^{(n + 1)}(x) &= \left[ \ln^{(x)} \right](x) \\
    &= (-1)^n n! x^{-n-1} 
\end{align*} 

\noindent Comme $\ln' > 0$ sur $\mathbb{R}_+^*$, alors $\ln$ est strictement croissante sur $\mathbb{R}_+^*$. \\

\section{Propriété fondamentale du logarithme}
On montre seulement la propriété pour $a > 0$ et $b > 0$. \\
On fixe $b > 0$ et on considère : 
\begin{align*}
    f: \mathbb{R}_+^* \rightarrow \mathbb{R} ; x \mapsto \ln (xb) \\
\end{align*}
Par composition, $f \in \mathcal{D}^1(\mathbb{R}_+^*, \mathbb{R})$ et :
\begin{align*}
    \forall x > 0, f'(x) = b \times \frac{1}{xb} = \frac{1}{x}
\end{align*}
Donc $f$ est une primitive de $\frac{1}{x}$ sur $\mathbb{R}_+^*$. \\
On choisit $c \in \mathbb{R}$ tel que : \\
\begin{align*}
    f = \ln + c
\end{align*}
En particulier : 
\begin{align*}
    f(1) = \ln 1 + c
\end{align*}
Soit : 
\begin{align*}
    \ln b = c
\end{align*}
Ainsi : 
\begin{align*}
    \forall x > 0, \ln (xb) = \ln x + \ln b
\end{align*}
On a par conséquent :
\begin{align*}
    \forall x \in \mathbb{R}_+^*, 0 &= \ln 1 \\
    &= \ln (x \times \frac{1}{x}) \\
    &= \ln x + \ln \frac{1}{x} \\
\end{align*}
Donc pour $a > 0$ et $b > 0$, on a :
\begin{align*}
    \ln \left( \frac{a}{b} \right) &= \ln \left( a \times \frac{1}{b} \right) \\
    &= \ln a + \ln \frac{1}{b} \\
    &= \ln a - \ln b
\end{align*}

\section{Limites usuelles de la fonction logarithme}
On commence par montrer que : 
\begin{align*}
    \ln x \underset{x \rightarrow +\infty}{\longrightarrow} +\infty
\end{align*}
On sait que $\ln$ est croissante sur $\mathbb{R}_+^*$, donc d'après le théorème de la limite monotone : 
\begin{align*}
    & \ln x \underset{x \rightarrow +\infty}{\longrightarrow} +\infty && \text{ou} && \ln x \underset{x \rightarrow +\infty}{\longrightarrow} \lambda
\end{align*}
Soit $n \geq 1$. On a :
\begin{align*}
    \ln n  &= \int_{1}^{n} \frac{\,dt}{t} \\
    &= \sum_{k = 1}^{n - 1} \int_{k}^{k + 1} \frac{\,dt}{t} \\
    &\geq \sum_{k = 1}^{n - 1} \int_{k}^{k + 1} \frac{\,dt}{k + 1} \\
    &= \sum_{k = 1}^{n - 1} \frac{1}{k + 1} \\
    &= \sum_{k = 1}^{n} \left( \frac{1}{k} \right) - 1 \\
\end{align*}
Or :
\begin{align*}
    \sum_{k=1}^{n} \left( \frac{1}{k} \right) - 1 \underset{n \rightarrow +\infty}{\longrightarrow} +\infty
\end{align*}
Par théorème de comparaison : 
\begin{align*}
    \ln n \underset{n \rightarrow +\infty}{\longrightarrow} +\infty
\end{align*}
Donc : 
\begin{align*}
    \ln x \underset{x \rightarrow +\infty}{\longrightarrow} +\infty
\end{align*} \\
Enfin : 
\begin{align*}
    \forall x > 0, \ln x = -\ln \left( \frac{1}{x} \right)
\end{align*}
Donc par composition :
\begin{align*}
    \ln x \underset{x \rightarrow 0^+}{\longrightarrow} -\infty
\end{align*} \\
Par taux d'accroissement, en introduisant : 
\begin{align*}
    f: \mathbb{R}_+ \rightarrow \mathbb{R} ; x \mapsto \ln (1 + x) \\
    f \in \mathcal{D}^1(\mathbb{R}_+, \mathbb{R}) \\
    \frac{\ln (x + 1)}{x} = \frac{f(x) - f(0)}{x - 0} \underset{x \rightarrow 0}{f'(0)} = 1
\end{align*}

\setcounter{section}{7}
\section{Propriétés de la fonction exponentielle}
D'après les résultas précédents $\text{(5.2)}$, $\text{(5.4)}$, on applique le théorème de la bijection dérivable. 
La fonction exponentielle est dérivable sur $\mathbb{R}$ et :
\begin{align*}
    \forall x \in \mathbb{R}, \exp' x &= \frac{1}{\ln' \circ \exp x} \\
    &= \exp x
\end{align*}
On obtient directement que $\exp \in \mathcal{C}^{\infty}(\mathbb{R}, \mathbb{R}_+^*)$ et que $\exp^{(n)} = \exp n$ pour tout $n \in \mathbb{N}$. \\

\section{Propriété fondamentale de l'exponentielle}
Soit $(x, y) \in \mathbb{R}^2$. On choisit $(a, b) \in (\mathbb{R}_+^*)^2$ tel que : 
\begin{align*}
    x = \ln a \text{ et } y = \ln b
\end{align*}
Ainsi : 
\begin{align*}
    \exp (x + y) &= \exp (\ln a + \ln b) \\
    &= \exp (\ln (ab)) \\
    &= ab \\
    &= \exp x \times \exp y
\end{align*}
Ainsi, $\exp 0 = \exp (0 + 0) = \exp^2 0$. \\
Donc $\exp 0 \in \{0 ; 1\}$ \\
Or $\exp$ est à valeur dans $\mathbb{R}_+^*$, donc $\exp 0 = 1$, donc :
\begin{align*}
    \forall x \in \mathbb{R}_+^*, \exp 0 = \exp (x - x) = \exp x \times \exp (-x) = 1
\end{align*}

\setcounter{section}{14}
\section{Dérivée d'une fonction puissance}
Soit $y > 0$. On pose $f:\mathbb{R} \to \mathbb{R} ; x \mapsto y^x = \exp (x \ln y)$. \\
$f \in \mathcal{D}^1(\mathbb{R}, \mathbb{R})$, donc par composition : 
\begin{align*}
    \forall x \in \mathbb{R}, f'(x) &= \ln y \times \exp (x \ln y) \\
    &= \ln y \times y^x
\end{align*}

\setcounter{section}{20}
\section{Croissances comparées en $+\infty$}
\begin{enumerate}
    \item On commence par montrer que $\frac{\ln x}{x} \underset{x \to +\infty}{\longrightarrow} 0$. \\
    Soit $x \geq 1$. On a : 
    \begin{align*}
        0 \leq \frac{\ln x}{x} &= \frac{1}{x} \int_{1}^{x} \frac{\,dt}{t} \\
        &\leq \frac{1}{x} \int_{1}^{x} \frac{\,dt}{\sqrt{t}} \\
        &= \frac{1}{x} \left[ 2 \sqrt{t} \right]_{1}^{x} \\
        &= \frac{2(\sqrt{x} - 1)}{x} \\
        &= 2 \left( \frac{1}{\sqrt{x}} - \frac{1}{x} \right) \\
        &\underset{x \to +\infty}{\longrightarrow} 0 
    \end{align*}
    D'après le théorème d'encadrement, $\frac{\ln x}{x} \underset{x \to +\infty}{\longrightarrow} 0$. \\
    Soit $a > 0$ et $x > 0$ :
    \begin{align*}
        &\frac{\ln x}{x^a} = \frac{1}{a} \times \frac{\ln x^a}{x^a} \underset{x \to +\infty}{\longrightarrow} 0 && \text{(composition et théorème d'opérations)} \\
    \end{align*}

    \item On utilise le changement de variable : 
    \begin{align*}
        x = (\ln y)^{\frac{1}{a}} \text{, soit } y = e^{ax}
    \end{align*}
    Ainsi : 
    \begin{align*}
        \frac{x^a}{e^x} &= \frac{\ln y}{y^{\frac{1}{a}}} \underset{x \to +\infty}{\longrightarrow} 
        \begin{cases}
            0 \text{ par composition si } a > 0 \\
            0 \text{ par théorème d'opérations si } a \leq 0 \\
        \end{cases}
    \end{align*}
\end{enumerate}

\section{Croissances comparées en $0$}
On utilise la proposition $\text{(5.21.1)}$ avec $y = \frac{1}{x}$. 

\setcounter{section}{43}
\setcounter{subsection}{1}
\subsection{Formule de trigonométrie hyperbolique}
Soit $(a, b) \in \mathbb{R}^2$. 
\begin{align*}
    ch(a)ch(b) + sh(a)sh(b) &= \frac{(e^a + e^{-a})(e^b + e^{-b})}{4} + \frac{(e^a - e^{-a})(e^b - e^{-b})}{4} \\
    &= \frac{2e^{a + b} + 2e^{-(a + b)}}{4} \\
    &= ch(a + b)
\end{align*}

\end{document}