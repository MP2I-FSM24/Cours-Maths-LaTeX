\documentclass[../main.tex]{subfiles}

\begin{document}
\setcounter{chapter}{15}
\chapter{Arithmétique des polynômes}
\tableofcontents
\clearpage

\section{Division euclidienne}
\begin{tcolorbox}[title=Théorème 16.1, title filled=false, colframe=orange, colback=orange!10!white]
    Soit $A \in \mathbb{K}[X]$ et $B \in \mathbb{K}[X]$ non nul, il existe un unique couple de polynômes $(Q,R)$ tel que $A = BQ + R$ avec $\deg R < \deg B$. Le polynôme $Q$ est appelé \textbf{quotient} et $R$ le \textbf{reste}. 
\end{tcolorbox}

\noindent \underline{Existence :} \\
On raisonne par récurrence sur le degré de $A$. \\
\begin{itemize}
    \item Pour $n = \deg A = 0$. Soit $A \in \mathbb{K}[X]$.
    \begin{itemize}
        \item Si $\deg B > 0$, alors $(0, A)$ convient. \\
        \item Si $\deg B = 0$, le couple $(B^{-1} \times A, 0)$ convient (comme $B$ est constant et non nul), alors $B \in \mathbb{K}^*$ donc inversible). \\
    \end{itemize}

    \item On suppose le résultat vrai pour tout $A \in \mathbb{K}_n[X]$. \\
    Soit $A \in \mathbb{K}_{n+1}[X]$ avec $\deg A = n+1$. \\
    On écrit $A = \underbrace{a}_{\neq 0} X^{n+1} + A_1$ avec $A_1 \in \mathbb{K}_n[X]$. 
    \begin{itemize}
        \item Si $\deg A < \deg B$, le couple $(0, A)$ convient. 
        \item Si $\deg A \geq \deg B$ et on note $b$ le coefficient dominant de $B$ : 
        \begin{align*}
            A - ab^{-1} B \times X^{n+1 - \deg B} \in \mathbb{K}_n[X]
        \end{align*}
        D'après l'hypothèse de récurrence, on choisit $(Q, R) \in \mathbb{K}[X]^2$ tel que $\deg R < \deg B$ et $A - ab^{-1} B \times X^{n+1 - \deg B} = QB + R$. \\
        Donc : 
        \begin{align*}
            A = \left[ Q + ab^{-1}X^{n+1 - \deg A} \right] \times B + R
        \end{align*}
    \end{itemize}
\end{itemize}

\noindent\underline{Unicité :} \\
On suppose que $A = BQ + R = BQ_1 + R_1$. \\
Donc : 
\begin{align*}
    B(Q - Q_1) &= R_1 - R \\
    \text{donc } \underbrace{\deg{(B(Q - Q_1))}}_{\deg{B} + \deg{Q - Q_1}} &= \deg{(R_1 - R)} \\
    &\leq \max(\deg{R_1}, \deg{R}) \\
    &< \deg{B} \\
    \text{donc } \deg{(Q - Q_1)} &< 0 \\
    \text{donc } Q - Q_1 &= 0 \\
    \text{puis } R_1 - R = 0
\end{align*}

\setcounter{section}{6}
\section{Proposition 16.7}
\begin{tcolorbox}[title=Propostion 16.7, title filled=false, colframe=lightblue, colback=lightblue!10!white]
    On a :
    \begin{enumerate}
        \item Soit $A$ et $P$ deux polynômes non nuls. Si $A|P$ et si $P|A$, alors il existe $\alpha \in \mathbb{K}^*$ tel que $P = \alpha A$. (La relation de divisibilité n'est pas antisymétrique)
        \item Si $A|B$ et si $B|C$, alors $A|C$. La relation de divisibilité est transitive. 
        \item Pour tout $A \in \mathbb{K}[X]$ non nul, $A|A$. La relation de divisibilité est réflexive. 
    \end{enumerate}
\end{tcolorbox}

\begin{enumerate}
    \item $P \neq 0$, $A \neq 0$. Si $A | P$ et $P | A$, alors (16.6.2) :
    \begin{align*}
        \deg A \leq \deg P \text{ et } \deg P \leq \deg A
    \end{align*}
    Donc : 
    \begin{align*}
        \deg P = \deg A
    \end{align*}
    Or $A|P$, alors : 
    \begin{align*}
        P = A \times Q
    \end{align*}
    Puis : 
    \begin{align*}
        \deg P = \deg (AQ) = \deg A + \deg Q \text{ ($\mathbb{K}$ est intègre)}
    \end{align*}
    Donc :
    \begin{align*}
        \deg Q = 0
    \end{align*}
    Donc : 
    \begin{align*}
        Q = \alpha \in \mathbb{K}^*
    \end{align*}

    \item RAS
    \item RAS
\end{enumerate}

\setcounter{section}{14}
\section{Principalité de $\mathbb{K}[X]$}
\begin{tcolorbox}[title=Théorème 16.15, title filled=false, colframe=orange, colback=orange!10!white]
    Soit $I$ un idéal de $\mathbb{K}[X]$ non réduit à $\{0\}$. Il existe un unique polynôme unitaire $D$ tel que
    $$I = D \mathbb{K}[X]$$
\end{tcolorbox}

\noindent \underline{Existence :} \\
Soit $I \neq \{0\}$ un idéal. \\
On note $A = \{ \deg P, P \in I\backslash \{0\} \} \subset \mathbb{N}$. \\
$A \neq \emptyset$ ($I \neq \{0\}$), d'après la propriété fondamentale de $\mathbb{N}$, $A$ possède un plus petit élément noté $n \geq 0$. \\
Comme $n \in A$, on choisit $D \in I$ tel que $\deg D = n$. \\
Comme $I$ est un idéal de $\mathbb{K}[X]$ et que $\mathbb{K} = \mathbb{K}_0[X] \subset \mathbb{K}[X]$, on a : 
\begin{align*}
    \forall \alpha \in \mathbb{K}, \alpha D \in I
\end{align*}
On peut donc supposer $D$ unitaire. 
Comme $I$ est un idéal de $\mathbb{K}[X]$, on a : 
\begin{align*}
    D \times \mathbb{K}[X] \subset I
\end{align*}
Soit $P \in I$. On effectue la division euclidienne de $P$ par $D$ ($\neq 0$) : 
\begin{align*}
    P = BD + R
\end{align*}
avec $\deg R \subset \deg D$. \\
Or : 
\begin{align*}
    R &= \underbrace{P}_{\in I} - \underbrace{BD}_{\in I} \\
    &\in I
\end{align*}
Par définition de $\deg D = n$, $R = 0$. \\ \\

\noindent \underline{Unicité :} \\
\begin{align*}
    I = D \mathbb{K}[X] = J \mathbb{K}[X] \\
\end{align*}
avec $D$ et $J$ unitaires. \\
Or ils sont associés, donc égaux. 

\setcounter{section}{16}
\section{Existence de $pgcd$}
\begin{tcolorbox}[title=Propostion 16.17, title filled=false, colframe=lightblue, colback=lightblue!10!white]
    Si $A$ et $B$ sont deux polynômes non nuls, de tels PGCD existent. 
\end{tcolorbox}

\noindent Soit $A, B$ dans $\mathbb{K}[X]$, $(A, B) \neq (0, 0)$.  \\
On note $\mathcal{C} = \{ \deg P, P|A \text{ et } P|B \text{ et } P\neq 0 \} \subset \mathbb{N}$. \\
$\mathcal{C} \neq \emptyset$ car $0 \in \mathcal{C}$ et $\mathcal{C}$ est majoré par $\deg B$ ($\max (\deg A, \deg B)$). \\
L'existence est assurée par la propriété fondamentale de $\mathbb{N}$. 

\section{Principalité de $\mathbb{K}[X]$}
\begin{tcolorbox}[title=Propostion 16.18, title filled=false, colframe=lightblue, colback=lightblue!10!white]
    Soit $A$ et $B$ deux polynômes non tous deux nuls. Soit $D \in \mathbb{K}[X]$. Alors $\Delta$ est un PGCD de $A$ et $B$ si et seulement si 
    $$A \mathbb{K}[X] + B \mathbb{K}[X] = D \mathbb{K}[X].$$
\end{tcolorbox}

\noindent D'après (16.15), on choisit $F \in \mathbb{K}[X]$ tel que :
\begin{align*}
    A \mathbb{K}[X] + B \mathbb{K}[X] = F \mathbb{K}[X]
\end{align*}
Soit $D \in \mathbb{K}[X]$. \\ \\

$\boxed{\Rightarrow}$ \\
On suppose que $D$ est un PGCD. \\
Donc $D|A$ et $D|B$. \\
Donc $D|F \text{ (combinaison $F \in A \mathbb{K}[X] + B \mathbb{K}[X]$)}$. \\
Or $F|A$ et $F|B$ ($A \in F \mathbb{K}[X]$, $B \in F \mathbb{K}[X]$). \\
Par maximalité de $\deg D$, on a $F$ et $D$ associés. \\ \\

$\boxed{\Leftarrow}$ \\
\begin{align*}
    D \mathbb{K}[X] = A \mathbb{K}[X] + B \mathbb{K}[X] = F \mathbb{K}[X]
\end{align*}
Donc $D|A$ et $D|B$. \\
Pour tout diviseur commun $P$ de $A$ et $B$, $P|A$ et $P|B$. \\
Donc $P|D$ ($D \in A \mathbb{K}[X] + B \mathbb{K}[X])$. \\
Donc $\deg D$ est maximal pour la divisibilité. 

\setcounter{section}{23}
\section{Lemme de préparation au calcul pratique du PGCD unitaire}
\begin{tcolorbox}[title=Lemme 16.24, title filled=false, colframe=orange, colback=orange!10!white]
    Soit $A$ et $B$ deux polynômes tels que $B \neq 0$. Pour tout $Q \in \mathbb{K}[X]$, on a $A \wedge B = (A - BQ) \wedge B$. \\
    En particulier, si $Q$ et $R$ sont le quotient et le reste de la division euclidienne de $A$ par $B$ Alors $A \wedge B = B \wedge R$.
\end{tcolorbox}

\begin{align*}
    (A \wedge B) \mathbb{K}[X] &= A \mathbb{K}[X] + B \mathbb{K}[X] \\
    &= (A - BQ) \mathbb{K}[X] + B \mathbb{K}[X] \\
    &= ((A - BQ) \wedge B) \mathbb{K}[X]
\end{align*}
Donc $A \wedge B$ et $(A - BQ) \wedge B$ sont associés, unitaires par définition, donc égaux. 

\setsection{25}
\section{Exemple}
\begin{tcolorbox}[title=Exemple alternatif 16.26, title filled=false, colframe=darkgreen, colback=darkgreen!10!white]
    Trouver les PGCD de $A = X^5 + 2X$ et de $B = X^4 + 2X^3 + 4$ et une relation de Bézout. 
\end{tcolorbox}

\begin{align*}
    X^5 + 2X &= (X^4 + 2X^3 + 4)(X - 2) + 4X^3 - 2X + 8 \\
    X^4 + 2X^3 + 4 &= (4X^3 - 2X + 8)(\frac{1}{4}X + \frac{1}{2}) + \frac{1}{2}X^2 - X \\
    4X^3 - 2X + 8 &= (\frac{1}{2}X^2 - X)(8X + 16) + 14X + 8 \\
    \frac{1}{2}X^2 - X &= (14X + 8)(\frac{1}{28}X - \frac{9}{14 \times 7}) + \frac{9 \times 4}{7^2} \\
    A \wedge B = 1
\end{align*}
\begin{align*}
    \frac{9 \times 4}{7^2} &= \frac{1}{2}X^2 - X - (14X + 8)(\frac{1}{28}X - \frac{9}{2 \times 7^2}) \\
    &= \frac{1}{2}X^2 - X - (4X^3 - 2X + 8 - (\frac{1}{2}X^2 - X)(8X + 16))(\frac{1}{28}X - \frac{9}{2 \times 7^2}) \\
\end{align*}

\section{Propriétés du PGCD}
\begin{tcolorbox}[title=Propostion 16.27, title filled=false, colframe=lightblue, colback=lightblue!10!white]
    L'opération $\wedge$ est commutative et associative. Par ailleurs, si $C$ est unitaire, alors $(A \wedge B) C = (AC) \wedge (BC)$.
\end{tcolorbox}

\noindent Soit $(A, B, C) \in \mathbb{K}[X]^3$ non tous nuls. 
\begin{align*}
    (A \wedge B) \mathbb{K}[X] &= A \mathbb{K}[X] + B \mathbb{K}[X] \\
    &= B \mathbb{K}[X] + A \mathbb{K}[X] \\
    &= (B \wedge A) \mathbb{K}[X]
\end{align*}
Donc $A \wedge B$ et $B \wedge A$ sont associés et unitaires donc égaux. 
\begin{align*}
    ((A \wedge B) \wedge C) \mathbb{K}[X] &= (A \wedge B) \mathbb{K}[X] + C \mathbb{K}[X] \\
    &= A \mathbb{K}[X] + B \mathbb{K}[X] + C \mathbb{K}[X] \\
    &= (A \wedge (B \wedge C)) \mathbb{K}[X]
\end{align*}
Donc $A \wedge (B \wedge C)$ et $(A \wedge B) \wedge C$ sont associés et unitaires donc égaux. \\
On suppose $C$ unitaire. \\
On a : 
\begin{align*}
    (A \wedge B) \mathbb{K}[X] &= A \mathbb{K}[X] + B \mathbb{K}[X] \\
    \text{donc } (A \wedge B)C \mathbb{K}[X] &= AC \mathbb{K}[X] + BC \mathbb{K}[X] \\
    &= ((AC) \wedge (BC)) \mathbb{K}[X]
\end{align*}
Ainsi $C(A \wedge B)$ et $(AC) \wedge (BC)$ sont associés et unitaires donc égaux.

\setsection{28}
\section{Existence de PPCM}
\begin{tcolorbox}[title=Propostion 16.29, title filled=false, colframe=lightblue, colback=lightblue!10!white]
    Soit $\mathbb{K}$ un corps. Soit $A$ et $B$ deux polynômes non nuls de $\mathbb{K}[X]$. Alors $A$ et $B$ admettent des PPCM. 
\end{tcolorbox}

\noindent On note $\mathcal{D} = \{ \deg P, A|P, B|P, P\neq 0 \} \subset \mathbb{N}$. \\
\begin{align*}
    \deg AB \in \mathcal{D} \neq \emptyset
\end{align*}
On conclut avec la propriété fondamentale de $\mathbb{N}$. 

\section{Caractérisation des PPCM par les idéaux}
\begin{tcolorbox}[title=Propostion 16.30, title filled=false, colframe=lightblue, colback=lightblue!10!white]
    Soit $A$ et $B$ deux polynômes non nuls de $\mathbb{K}[X]$ et soit $P \in \mathbb{K}[X]$. Alors $P$ est un PPCM de $A$ et $B$ si et seulement si 
    $$A \mathbb{K}[X] \cap B \mathbb{K}[X] = P \mathbb{K}[X].$$
\end{tcolorbox}

\noindent $A \mathbb{K}[X] \cap B \mathbb{K}[X]$ est un idéal de $\mathbb{K}[X]$, donc de la forme $M \mathbb{K}[X]$ (16.15). \\
Montrons que $P$ est un PPCM de $A$ et $B$ si et seulement si $P$ et $M$ sont associés. \\ \\

$\boxed{\Rightarrow}$ \\
On a donc : 
\begin{align*}
    P &\in A \mathbb{K}[X] \cap B \mathbb{K}[X] \\
    &\in M \mathbb{K}[X] \\
\end{align*}
Donc $M|P$. \\
Or $M$ est un multiple commun à $A$ et $B$, donc par définition de $P$, on a : 
\begin{align*}
    \boxed{\deg P \leq \deg M}
\end{align*}
Donc $P$ et $M$ sont associés. \\ \\

$\boxed{\Leftarrow}$ \\
On suppose $P$ et $M$ associés, donc : 
\begin{align*}
    P \mathbb{K}[X] &= M \mathbb{K}[X] \\
    &= A \mathbb{K}[X] \cap B \mathbb{K}[X]
\end{align*}
En particulier, $P$ est un multiple commun à $A$ et $B$ et pour tout $Q \in A \mathbb{K}[X] \cap B \mathbb{K}[X]$, donc $P|Q$. \\
Donc : 
\begin{align*}
    \boxed{deg P \leq \deg Q}
\end{align*}


\end{document}