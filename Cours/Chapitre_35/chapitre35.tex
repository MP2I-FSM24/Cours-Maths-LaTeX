\documentclass[../main.tex]{subfiles}

\begin{document}
\setcounter{chapter}{34}
\chapter{Familles sommables}
\tableofcontents
\clearpage

\setsection{1}
\section{Reformulation}
\begin{tcolorbox}[title=Propostion 35.2, title filled=false, colframe=lightblue, colback=lightblue!10!white]
    Soit $\sum a_n$ une séries à termes positifs. Alors $\sum_{n \geq 0} a_n$ est bien définie dans $\overline{\mathbb{R}}_{+}$et
    $$\sum_{n \geq 0} a_n=\sup \left\{\sum_{k \in J} a_k, J \in \mathcal{P}_f(\mathbb{N})\right\}$$
\end{tcolorbox}

\noindent En notant, pour $n\in \mathbb{N}, S_n = \sum\limits_{k=0}^{n} a_k$, on a : 
\begin{align*}
    S_n \underset{n \to +\infty}{\longrightarrow} \sum_{k\geq 0}^{a_k} 
\end{align*}
Or pour tout $n\in \mathbb{N}, S_n\in \left\{ \sum_{k\in J} a_k\mid J\in \mathcal{P}_f(\mathbb{N}) \right\}$. \\
Donc $\sum_{k\geq 0} a_k \leq \sup \left\{ \sum_{k\in J} a_k\mid J\in \mathcal{P}_f(\mathbb{N}) \right\} = S$. \\
Par ailleurs, pour $J\in \mathcal{P}_f(\mathbb{N})$, on pose $N = \max J$ et $J\subset \llbracket 0, N \rrbracket$ et : 
\begin{align*}
    \sum_{k\in J} a_k  \leq \sum_{k=0}^{N} a_k \leq \sum_{k>0} a_k 
\end{align*}
Par définition de la borne supérieure : 
\begin{align*}
    S \leq \sum_{k\geq 0} a_k
\end{align*}
Donc : 
\begin{align*}
    \sum_{k\geq 0} a_k = S
\end{align*}

\setsection{4}
\section{Croissance de la somme}
\begin{tcolorbox}[title=Propostion 35.5, title filled=false, colframe=lightblue, colback=lightblue!10!white]
    Soit $\left(a_i\right)_{i \in I}$ et $\left(b_i\right)_{i \in I}$ deux familles à valeurs dans $\overline{\mathbb{R}}_{+}$. Si pour tout $i \in I, a_i \leq b_i$, alors
    $$\sum_{i \in I} a_i \leq \sum_{i \in I} b_i$$
\end{tcolorbox}

\noindent Soit $J\in \mathcal{P}_f(I)$. Comme : 
\begin{align*}
    \forall i\in J, a_i\leq b_i
\end{align*}
Alors : 
\begin{align*}
    \sum_{i\in j} a_i \leq \sumç{i\in J} b_i \leq \sum_{i\in I} b_i
\end{align*}
$\sum\limits_{i\in I} b_i$ est un majorant de $\left\{ \sum_{i\in J} a_i\mid J\in \mathcal{P}_f(I) \right\}$. \\
Par définition : 
\begin{align*}
    \sum_{i\in I} a_i \leq \sum_{i\in I} b_i
\end{align*}

\setsection{7}
\section{Lien avec les séries à termes positifs}
\begin{tcolorbox}[title=Propostion 35.8, title filled=false, colframe=lightblue, colback=lightblue!10!white]
    Soit $\sum a_n$ une séries à termes positifs.
    \begin{enumerate}
        \item On a $\sum_{n=0}^{+\infty} a_n=\sum_{n \in \mathbb{N}} a_n$ (le terme de gauche correspond à la somme de la série tandis que le terme de droite à la somme de la famille sommable). 
        \item En particulier, $\sum a_n$ converge si et seulement si la famille $\left(a_n\right)_{n \in \mathbb{N}}$ est sommable. 
    \end{enumerate}
\end{tcolorbox}

\begin{enumerate}
    \item (35.2)
    \item Théorème de la Limite Monotone
\end{enumerate}

\setsection{9}
\section{Invariance de la somme d'une série à termes positifs par permutation des termes}
\begin{tcolorbox}[title=Corollaire 35.10, title filled=false, colframe=orange, colback=orange!10!white]
    Soit $\sum a_n$ une série à termes positifs et $\sigma \in S_{\mathbb{N}}$. Alors
    $$\sum_{n=0}^{+\infty} a_n=\sum_{n=0}^{+\infty} a_{\sigma(n)}$$
    Cette égalité reste vraie dans $\overline{\mathbb{R}}_{+}$.
\end{tcolorbox}

\noindent Soit $\sigma\in \mathcal{S}_\mathbb{N}$. \\
$\sigma$ induit une bijection $\mathcal{P}_f(\mathbb{N})\to \mathcal{P}_f(\mathbb{N});J\mapsto \sigma(J)$ de $\left\{ \sum_{i\in J}| J\in \mathcal{P}_f(\mathbb{N}) \right\}$ sur $\left\{ \sum_{i\in I} a_{\sigma(i)}\mid J\in \mathcal{P}_f(\mathbb{N}) \right\}$. \\
Ces deux ensembles ont donc la même borne supérieure. Donc : 
\begin{align*}
    \sum_{i\in \mathbb{N}} a_i = \sum_{i\in \mathbb{N}} a_{\sigma(i)} 
\end{align*}
Soit : 
\begin{align*}
    \sum_{i=0}^{+\infty} a_i = \sum_{i=0}^{+\infty} a_{\sigma(i)} \text{ (35.8)}
\end{align*}


\end{document}