\documentclass[../main.tex]{subfiles}

\begin{document}
\setcounter{chapter}{34}
\chapter{Familles sommables}
\tableofcontents
\clearpage

\setsection{1}
\section{Reformulation}
\begin{tcolorbox}[title=Propostion 35.2, title filled=false, colframe=lightblue, colback=lightblue!10!white]
    Soit $\sum a_n$ une séries à termes positifs. Alors $\sum_{n \geq 0} a_n$ est bien définie dans $\overline{\mathbb{R}}_{+}$et
    $$\sum_{n \geq 0} a_n=\sup \left\{\sum_{k \in J} a_k, J \in \mathcal{P}_f(\mathbb{N})\right\}$$
\end{tcolorbox}

\noindent En notant, pour $n\in \mathbb{N}, S_n = \sum\limits_{k=0}^{n} a_k$, on a : 
\begin{align*}
    S_n \underset{n \to +\infty}{\longrightarrow} \sum_{k\geq 0}^{a_k} 
\end{align*}
Or pour tout $n\in \mathbb{N}, S_n\in \left\{ \sum_{k\in J} a_k\mid J\in \mathcal{P}_f(\mathbb{N}) \right\}$. \\
Donc $\sum_{k\geq 0} a_k \leq \sup \left\{ \sum_{k\in J} a_k\mid J\in \mathcal{P}_f(\mathbb{N}) \right\} = S$. \\
Par ailleurs, pour $J\in \mathcal{P}_f(\mathbb{N})$, on pose $N = \max J$ et $J\subset \llbracket 0, N \rrbracket$ et : 
\begin{align*}
    \sum_{k\in J} a_k  \leq \sum_{k=0}^{N} a_k \leq \sum_{k>0} a_k 
\end{align*}
Par définition de la borne supérieure : 
\begin{align*}
    S \leq \sum_{k\geq 0} a_k
\end{align*}
Donc : 
\begin{align*}
    \sum_{k\geq 0} a_k = S
\end{align*}


\end{document}