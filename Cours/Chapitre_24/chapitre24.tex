\documentclass[../main.tex]{subfiles}

\begin{document}
\setcounter{chapter}{23}
\chapter{Comparaison locale des suites}
\tableofcontents
\clearpage

\setsection{17}
\section{Caractérisation de l'équivalence par la négligabilité}
\begin{tcolorbox}[title=Propostion 24.18, title filled=false, colframe=lightblue, colback=lightblue!10!white]
    On a : 
    \begin{align*}
        u_n \sim v_n \Leftrightarrow u_n = v_n + o(v_n)
    \end{align*}
\end{tcolorbox}

$\boxed{\Rightarrow}$ \\
Si $u_n \sim v_n$ à partir d'un certain rang : 
\begin{align*}
    u_n = a_n v_n \text{ avec } a_n \underset{n \to +\infty}{\longrightarrow} 1
\end{align*}
Ainsi : 
\begin{align*}
    u_n &= \underbrace{(a_n - 1)}_{= o(1)} v_n + v_n \\
    &=_{n\to +\infty} v_n + o(v_n)
\end{align*} \\

$\boxed{\Leftarrow}$ \\
Si $u_n = v_n + o(v_n)$, alors à partir d'un certain rang :
\begin{align*}
    u_n &= v_n + \epsilon_n v_n \text{ avec } \epsilon_n = o(1) \\
    &= \underbrace{(1 + \epsilon_n)}_{\underset{n \to +\infty}{\longrightarrow} 1} v_n
\end{align*}
Donc : 
\begin{align*}
    u_n \sim v_n
\end{align*}

\setsection{19}
\section{Equivalent d'un polynôme}
\begin{tcolorbox}[title=Propostion 24.20, title filled=false, colframe=lightblue, colback=lightblue!10!white]
    Soit $P$ un polynôme de monôme dominant $a_d X^d$. Alors $P(n) \sim a_d n^d$.
\end{tcolorbox}

\noindent On note $P = \sum\limits_{k=0}^d a_k X^k$. \\
Pour $k\in \llbracket 0, d-1 \rrbracket$ : 
\begin{align*}
    n^k =_{n\to +\infty} o(n^d) \text{ et } a_k n^k =_{n\to +\infty} o(a_d n^d)
\end{align*}
Donc :
\begin{align*}
    \sum_{k=0}^{d-1} a_k n^k =_{n\to +\infty} o(a_d n^d)
\end{align*}
Donc : 
\begin{align*}
    P(n) &= a_d n^d + o(a_d n^d) \\
    &\sim a_d n^d
\end{align*}


\end{document}