\documentclass[../main.tex]{subfiles}

\begin{document}
\setcounter{chapter}{31}
\chapter{Espaces probabilisés finis}
\tableofcontents
\newpage

\setsection{18}
\section{Exemple}
\begin{tcolorbox}[title=Exemple 32.19, title filled=false, colframe=darkgreen, colback=darkgreen!10!white]
    Une urne contient $3$ boules blanches et $5$ boules noires. On en tire simultanément $4$ boules. Avec quelle probabilité n'a-t-on tiré que des boules noires ?
\end{tcolorbox}

\noindent Sans perte de généralité, on peut numéroter les boules de $1$ à $8$, les $3$ premières boules sont blanches et les $5$ suivantes noires. \\
On note $X$ la variable alétoire donnant la $4$-combinaison des boules obtenues. \\
$$X\hookrightarrow \mathcal{U}(P_4 \llbracket 1, 8 \rrbracket)$$
En notant $A$ "on ne tire que des boules noires", on a :
\begin{align*}
    A &= (X\in P_4 \llbracket 4, 8 \rrbracket) \\
    P(A) &= P(X\in P_4 \llbracket 4, 8 \rrbracket) \\
    &= \frac{|P_4 \llbracket 4, 8 \rrbracket|}{|P_4 \llbracket 1, 8 \rrbracket|} \\
    &= \frac{\binom{5}{4}}{\binom{8}{4}}
\end{align*}


\end{document}