\documentclass[../main.tex]{subfiles}

\begin{document}
\setcounter{chapter}{30}
\chapter{Dénombrement}
\tableofcontents
\clearpage

\setsection{11}
\section{Exemple : parcours d'une fourmi}
\begin{tcolorbox}[title=Exemple 31.12, title filled=false, colframe=darkgreen, colback=darkgreen!10!white]
    La fourmi Donald se promène sur un grillage du plan de taille $2\times p$ dont chaque arête est de longueur $1$. Combien de chemins de longueur minimale peut-elle emprunter pour gagner le point d'arrivée depuis son point de départ ?
\end{tcolorbox}

\noindent Compter ce nombre de chemins revient à dénombrer le nombre de mots de $p+2$ lettres contenant exactement $p$ lettres $D$ et $2$ lettres $B$. \\
Pour constuire un tel mot, il suffit de choisir la place des deux $B$. \\
On a $p+1$ choix pour le premier $B$. \\
Pour chaque choix de position $k\in \llbracket 1, p+1\rrbracket$, il reste $p+2-k$ choix pour le second $B$. \\
Le nombbre de choix possible final est donc : 
\begin{align*}
    \sum_{k=1}^{p+1} (p+2-k) &= \sum_{k=1}^{p+1} k \\
    &= \frac{(p+1)(p+2)}{2} 
\end{align*}

\setsection{18}
\section{Exemple}
\begin{tcolorbox}[title=Exemple 31.19, title filled=false, colframe=darkgreen, colback=darkgreen!10!white]
    Combien y-a-t-il de couples $(x, y)$ dans $\llbracket 1, n \rrbracket^2$ avec $x\neq y$ ?
\end{tcolorbox}
\noindent\underline{Etape 1 :} On choisit $x\in \llbracket 1, n \rrbracket$, soit $n$ choix. \\
\noindent\underline{Etape 2 :} On choisit $y\in \llbracket 1, n \rrbracket\setminus\{x\}$, soit $n-1$ choix. \\
Au total $n(n-1)$ choix (principe des bergers). 

\section{Exemple}
\begin{tcolorbox}[title=Exemple 31.20, title filled=false, colframe=darkgreen, colback=darkgreen!10!white]
    A partir d'un alphabet de $p$ lettres, combien de mots de $n$ lettres peut-on former qui ne contiennent jamais deux lettres identiques consécutives ?
\end{tcolorbox}

\noindent\underline{Etape 1 :} On choisit la première lettre : $p$ possibilités. \\
\noindent\underline{Etape 2 :} On choisit la deuxième lettre : $p-1$ possibilités. \\
\noindent\underline{Etape 3 :} On choisit la troisième lettre : $p-1$ possibilités. \\
...\\
Au total : $p(p-1)^{n-1}$ possibilités. 

\setsection{26}
\section{Exemple}
\begin{tcolorbox}[title=Exemple 31.27, title filled=false, colframe=darkgreen, colback=darkgreen!10!white]
    De combien de façon peut-on tirer $5$ cartes successivement avec remise dans un jeu de $52$ cartes ?
\end{tcolorbox}

\noindent Il s'agit de compter le nombre de $5$-listes d'un ensemble de cardinal $52$, soit $52^5$ possibilités. 

\section{Exemple}
\begin{tcolorbox}[title=Exemple 31.28, title filled=false, colframe=darkgreen, colback=darkgreen!10!white]
    Combien y-a-t-il de mots de $7$ lettres contenant le mot "OUPS" ?
\end{tcolorbox}

\noindent\underline{Etape 1 :} Choix de la place du mot "OUPS" : $4$ choix possibles. \\
\noindent\underline{Etape 2 :} On complète avec un mot de $3$ lettres. \\ Cela revient à compter le nombre de $3$-listes d'un ensemble à $26$ éléments : $26^3$ possibilités. \\
Aut total : $4\times 26^3$ possibilités.

\setsection{31}
\section{Exemple}
\begin{tcolorbox}[title=Exemple 31.32, title filled=false, colframe=darkgreen, colback=darkgreen!10!white]
    De combien de façon peut-on tirer $5$ cartes successivement sans remise dans un jeu de $52$ cartes ?
\end{tcolorbox}

\noindent Cela revient à compter le nombre de $5$-arrangements d'un ensemble de cardinal $52$, soit $\frac{52!}{(52-5)!}$. 

\section{Exemple}
\begin{tcolorbox}[title=Exemple 31.33, title filled=false, colframe=darkgreen, colback=darkgreen!10!white]
    De combien de façons peut-on asseoir $n$ personnes sur un banc rectiligne ? Autour d'une table ronde ?
\end{tcolorbox}

\begin{itemize}
    \item Sur un banc rectiligne, cela revient à calculer le nombre de $n$-arrangements d'un ensemble de cardinal $n$, soit $n!$ choix.
    \item En choisissant arbitrairement la place d'une personne (par exemple Jack), il suffit de compléter par un $(n-1)$-arrangement d'un ensemble à $n-1$ éléments, soit $(n-1)!$ choix. 
\end{itemize}


\end{document}