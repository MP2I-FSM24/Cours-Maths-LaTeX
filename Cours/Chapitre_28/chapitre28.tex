\documentclass[../main.tex]{subfiles}

\begin{document}
\setcounter{chapter}{27}
\chapter{Matrice d'une application linéaire}
\tableofcontents
\clearpage

\setsection{4}
\section{Interprétation vectorielle de l'inversibilité, cas des familles de vecteurs}
\begin{tcolorbox}[title=Théorème 28.5, title filled=false, colframe=orange, colback=orange!10!white]
    Soit $E$ un $\mathbb{K}$-ev de dimension finie $n\neq 0$, $\mathcal{B}$ une base de $E$, $\mathcal{F}$ une famille de $n$ vecteurs de $E$. Alors $\mathcal{F}$ est une base de $E$ si et seulement si $Mat_{\mathcal{B}}(\mathcal{F})$ est inversible.
\end{tcolorbox}

\noindent Soit $\mathcal{F} = (x_1, \ldots, x_n)$ une famille de vecteurs et $\mathcal{B} = (b_1, \ldots, b_n)$ une base de $E$. \\
On note $M = Mat_{\mathcal{B}}(\mathcal{F}) = (m_{ij})_{1\leq i,j\leq n}$. Ainsi :
\begin{align*}
    \forall j\in \llbracket 1, n \rrbracket, x_j = \sum_{i=1}^n m_{ij} b_i
\end{align*}
$\mathcal{F}$ est une base de $E$ si et seulement si $\mathcal{F}$ est libre (car $|\mathcal{F}| = \dim E$), si et seulement si : 
\begin{align*}
    \forall (\lambda_1, \ldots, \lambda_n)\in \mathbb{K}^n, \sum_{j=1}^n \lambda_j x_j = 0 \Rightarrow \forall j\in \llbracket 1, n \rrbracket, \lambda_j = 0
\end{align*}
Or pour $(\lambda_1, \ldots, \lambda_n)\in \mathbb{K}^n$ : 
\begin{align*}
    \sum_{j=1}^{n} \lambda_j x_j &= \sum_{j=1}^n \lambda_j \sum_{i=1}^n m_{ij} b_i \\
    &= \sum_{i=1}^{n} \left( \sum_{j=1}^n m_{ij} \lambda_j \right) b_i \\
    &= \sum_{i=1}^{n} \left[ M \begin{pmatrix}
        \lambda_1 \\
        \vdots \\
        \lambda_n
    \end{pmatrix} \right]_i b_i
\end{align*}
Ainsi : 
\begin{align*}
    \sum_{j=1}^{n} \lambda_j x_j = 0 &\Leftrightarrow \left[ \forall i\in \llbracket 1, n \rrbracket, \left[ M \begin{pmatrix}
        \lambda_1 \\
        \vdots \\
        \lambda_n
    \end{pmatrix} \right]_i = 0 \right] \\
    &\Leftrightarrow M \begin{pmatrix}
        \lambda_1 \\
        \vdots \\
        \lambda_n
    \end{pmatrix} = \begin{pmatrix}
        0 \\
        \vdots \\
        0
    \end{pmatrix} \\
    &\Leftrightarrow \begin{pmatrix}
        \lambda_1 \\
        \vdots \\
        \lambda_n
    \end{pmatrix} \in \ker M
\end{align*}
En conclusion, $\mathcal{F}$ est une base si et seulement si $\ker M = \{0\}$, si et seulement si $M$ est inversible. 

\section{Exemple}
\begin{tcolorbox}[title=Exemple 28.6, title filled=false, colframe=darkgreen, colback=darkgreen!10!white]
    Montrer que la famille $(X^2 + 3X + 1, 2X^2 + X, x^2)$ de $\mathbb{R}[X]$ est libre. 
\end{tcolorbox}

\noindent On note $\mathcal{B} = (1, X, X^2)$. \\
$Mat_{\mathcal{B}}(\mathcal{F}) = \begin{pmatrix}
    1 & 0 & 0 \\
    3 & 1 & 0 \\
    1 & 2 & 1
\end{pmatrix}$ est triangulaire inférieure avec une diagonale ne contenant aucun $0$ : elle est donc inversible. Donc $\mathcal{F}$ est une base de $\mathbb{R}_2[X]$, donc libre. 


\end{document}