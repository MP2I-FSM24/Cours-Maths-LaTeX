\documentclass[../main.tex]{subfiles}

\begin{document}
\setcounter{chapter}{0}
\chapter{Calculs Algébriques}
\tableofcontents
\clearpage

\setcounter{section}{19}
\section{Somme des carrés et des cubes}
\begin{itemize}
    \item Somme des carrés : \\
    Pour tout $n \in \mathbb{N}$, on note la proposition :
    $$P(n): \text\og\sum_{k=1}^{n}k^2=\frac{n(n+1)(2n+1)}{6}\text\fg$$
    Démontrons-la par récurrence.\\
    \\
    \underline{Initialisation :} Pour $n=0$, on a :
    $$\sum_{k=1}^{0}k^2=0$$
    et :
    $$\frac{0\times(0+1)\times(2\times0+1)}{6}=0$$
    Donc $P(0)$ est vraie.\\
    \\
    \underline{Hérédité :} On suppose $P(n)$ vraie pour un $n$ fixé dans
    $\mathbb N$. On a :
    \begin{align*}
    \sum_{k=1}^{n+1}k^2 & = \sum_{k=1}^{n}k^2 + (n+1)^2 \\
    & = \frac{n(n+1)(2n+1)}{6} + (n+1)^2 \\
    & = \frac{n+1}{6}(n(2n+1) + 6(n+1)) \\
    & = \frac{n+1}{6}(2n^2 + 7n + 6) \\
    & = \frac{(n+1)(n+2)(2n+3)}{6}
    \end{align*}
    Donc $P(n+1)$ est vraie aussi.\\
    \\
    \underline{Conclusion :} D'après le principe de récurrence,
    $$\forall n \in \mathbb{N}, \sum_{k=1}^{n}k^2=\frac{n(n+1)(2n+1)}{6}$$
    \\
    \item Somme des cubes : \\
    Pour tout $n \in \mathbb{N}$, on note la proposition :
    $$P(n): \text\og\sum_{k=1}^{n}k^3=\frac{n^2(n+1)^2}{4}\text\fg$$
    Démontrons-la par récurrence.\\
    \\
    \underline{Initialisation :} Pour $n=0$, on a :
    $$\sum_{k=1}^{0}k^3=0$$
    et :
    $$\frac{0\times(0+1)^2}{4}=0$$
    Donc $P(0)$ est vraie.\\
    \\
    \underline{Hérédité :} On suppose $P(n)$ vraie pour un $n$ fixé dans
    $\mathbb N$. On a :
    \begin{align*}
    \sum_{k=1}^{n+1}k^3 & = \sum_{k=1}^{n}k^3 + (n+1)^3 \\
    & = \frac{n^2(n+1)^2}{4} + (n+1)^3 \\
    & = \frac{(n+1)^2}{4}(n^2 + 4(n+1)) \\
    & = \frac{(n+1)^2}{4}(n^2 + 4n + 4) \\
    & = \frac{(n+1)^2(n+2)^2}{4}
    \end{align*}
    Donc $P(n+1)$ est vraie aussi.\\
    \\
    \underline{Conclusion :} D'après le principe de récurrence,
    $$\forall n \in \mathbb{N}, \sum_{k=1}^{n}k^2=\frac{n^2(n+1)^2}{4}$$
\end{itemize}

\setcounter{section}{38}
\section{Formule de Pascal}
Démontrons pour tout $(n,p)\in(\mathbb{N}^*)^2$ la relation :
$$\text\og\binom{n}{p}=\binom{n-1}{p-1}+\binom{n-1}{p}\text\fg$$
La relation est vraie si $p>n$ (on a $0 = 0+0$) et si $p=n$ (qui
donne $1=0+1$).\\
\\
Soit $1\le p\le n$ :
\begin{align*}
  \binom{n-1}{p}+\binom{n-1}{p-1} & = \frac{(n-1)!}{p!(n-1-p)!} +
  \frac{(n-1)!}{(p-1)!(n-p)!} \\[8pt]
  & = \frac{(n-1)!}{(p-1)!(n-1-p)!}\bigg(\frac 1 p + \frac{1}{n-p}\bigg) \\[8pt]
  & = \frac{(n-1)!\times n}{(p-1)!(n-1-p)!\times p(n-p)} \\[8pt]
  & = \frac{n!}{p!(n-p)!} \\[8pt]
  & = \binom{n}{p}
\end{align*}

\setcounter{section}{40}
\section{Formule du capitaine}
Démontrons pour $n$ et $p$ deux entiers tels que $1\le p\le n$ la relation :
$$\text\og n\binom{n-1}{p-1}=p\binom{n}{p}\text\fg$$
On a :
$$n\binom{n-1}{p-1}=n\times\frac{(n-1)!}{(p-1)!(n-p)!}=p\times\frac{n!}{p!(n-p)!}=p\binom{n}{p}$$

\section{Formule du binôme de Newton}
Soit $(x,y)\in\mathbb{C}^2$. Pour tout $n \in \mathbb{N}$, on note la
proposition :
$$P(n): \text\og(x+y)^n=\sum_{k=0}^{n}x^k y^{n-k}\text\fg$$
Démontrons-la par récurrence.\\
\\
\underline{Initialisation :} Pour $n=0$, on a :
$$(x+y)^0=1$$
et
$$\sum_{k=0}^{0}\binom{0}{k}x^k y^{0-k}=\binom{0}{0}x^0 y^0=1$$
Donc $P(0)$ est vraie.\\
\\
\underline{Hérédité :} On suppose $P(n)$ vraie pour un $n$ fixé dans
$\mathbb N$. On a :
\begin{align*}
  (x+y)^{n+1} & = (x+y)(x+y)^n \\
  & = (x+y)\sum_{k=0}^{n}\binom{n}{k}x^k y^{n-k} &&
  \textit{(hypothèse de récurrence)}\\
  & = \sum_{k=0}^{n}\binom{n}{k}(x^{k+1}y^{n-k}+x^k y^{n+1-k}) &&
  \textit{(linéarité)}\\
  & =
  \sum_{k=0}^{n}\binom{n}{k}x^{k+1}y^{n-k}+\sum_{k=0}^{n}\binom{n}{k}x^k
  y^{n+1-k}\\
  & = \sum_{k=1}^{n+1}\binom{n}{k-1}x^k y^{n-k} +
  \sum_{k=0}^{n}\binom{n}{k}x^k y^{n+1-k} && \textit{(translation)}\\
  & = x^{n+1} + \sum_{k=1}^{n}x^k
  y^{n+1-k}\Bigg(\binom{n}{k-1}+\binom{n}{k}\Bigg)+y^{n+1}\\
  & = x^{n+1} + \sum_{k=1}^{n}\binom{n+1}{k}x^k y^{n+1-k}+y^{n+1} &&
  \textit{(formule de Pascal)}\\
  & = \sum_{k=0}^{n+1}x^k y^{n+1-k}
\end{align*}
Donc $P(n+1)$ est vraie aussi.\\
\\
\underline{Conclusion :} D'après le principe de récurrence,
$$\forall n \in \mathbb{N}, (x+y)^n=\sum_{k=0}^{n}\binom{n}{k}x^k y^{n-k}$$

\end{document}