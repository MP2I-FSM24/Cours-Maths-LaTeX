\documentclass[../main.tex]{subfiles}

\begin{document}
\setcounter{chapter}{17}
\chapter{Dérivabilité}
\tableofcontents
\clearpage

\setsection{12}
\section{Condition nécessaire du premier ordre pour l'existence d'un extremum}
\begin{tcolorbox}[title=Théorème 18.13, title filled=false, colframe=orange, colback=orange!10!white]
    Soit $f$ une fonction définie sur $I$ un intervalle ouvert et $x_0 \in I$. Si $f$ est dérivable en $x_0$ et admet un extremum local en $x_0$, alors $f'(x_0) = 0$. 
\end{tcolorbox}

\noindent On suppose que $f$ atteint un maximum local en $x_0$. \\
On choisit $U \in \mathcal V(x_0)$ tel que : 
\begin{align*}
    \forall x \in U \cap I, f(x) \leq f(x_0)
\end{align*}
En particulier : 
\begin{align*}
    \forall x \in U, x > x_0, \frac{f(x) - f(x_0)}{x - x_0} &\leq 0 \\
    \forall x \in U, x < x_0, \frac{f(x_0) - f(x)}{x_0 - x} &\geq 0
\end{align*}
D'après le TCILPPL : 
\begin{align*}
    f'_{\text{droite}}(x_0) \leq 0 \text{ et } f'_{\text{gauche}}(x_0) \geq 0
\end{align*}
Donc $f$ est dérivable en $x_0$. \\
Donc $f'_g(x_0) = f'_d(x_0) = 0$. 

\setsection{42}
\section{Théorème de prolongement de classe $\mathcal{C}^n$ - HP}
\begin{tcolorbox}[title=Théorème 18.43 - HP, title filled=false, colframe=orange, colback=orange!10!white]
    Soit $I$ un intervalle et $x_0 \in I$. Soit $f$ une fonction définie de classe $\mathcal{C}^n$ sur $I\backslash \{x_0\}$. Si $f^{(n)}$ admet une limite finie en $x_0$, alors $f$ est prolongeable en une fonction de classe $\mathcal{C}^n$ sur $I$.
\end{tcolorbox}

\begin{itemize}
    \item On prouve le théorème pour $n=1$. On suppose $f \in \mathcal{C}^1(I\backslash \{x_0\}, \mathbb{R})$ et que $f'$ admet une limite finie en $x_0$. \\
    On prolonge $f'$ en une fonction $g$ par continuité en $x_0$. Ainsi, $g \in \mathcal{C}^0(I, \mathbb{R})$. \\
    On remarque que pour tout $x \neq x_0$ : 
    \begin{align*}
        f(x) = f(a) + \int_a^x f'(t) \, dt
    \end{align*}
    où $a \in I\backslash \{x_0\}$ quelconque. 
    \begin{align*}
        f(x) &= \underbrace{f(a) + \int_a^x g(t) \, dt}_{\text{Admet une limite finie quand } x\to x_0} \\
    \end{align*}
    Donc $f(x)$ admet également une limite finie quand $x\to x_0$. \\
    On prolonge alors $f$ par continuité en $\tilde{f}$, de classe $\mathcal{C}^1$ sur $I$. \\

    \item On raisonne par récurrence. Pour $n \in \mathbb{N}$, on pose : 
    \begin{align*}
        P(n) : "\text{Pour tout $f \in \mathcal{C}^n(I\backslash \{x_0\}, \mathbb{R})$, si $f^{(n)}$ admet une limite finie en $x_0$, alors $f$ se prolonge en $\tilde{f} \in \mathcal{C}^n(I, \mathbb{R})$}".
    \end{align*}
    Pour $n = 0$, c'est le prolongement par continuité. \\
    Pour $n = 1$, c'est fait. \\
    On suppose $P(n)$ vraie pour $n \geq 1$. \\
    Soit $f\in \mathcal{C}^{n+1}(I\backslash \{x_0\}, \mathbb{R})$, etc... \\
    Donc $f'\in \mathcal{C}^n(I\backslash \{x_0\}, \mathbb{R})$ et $f^{(n)}$ admet une limite finie en $x_0$. \\
    D'après $P(n)$, on prolonge $f'$ en $g\in \mathcal{C}^n(I, \mathbb{R})$. \\
    En particulier, $g$ est continue sur $I$. \\
    Donc $f'$ admet une limite finie en $x_0$. \\
    On applique $P(1)$. On prolonge $f$ en $\tilde{f}\in \mathcal{C}^{n+1}(I, \mathbb{R})$. \\
    Or $\tilde{f'} = g \in \mathcal{C}^n(I, \mathbb{R})$. \\
    Donc $\tilde f \in \mathcal{C}^{n+1}(I, \mathbb{R})$. \\
\end{itemize}

\setsection{44}
\section{IAF pour les fonctions à valeurs dans $\mathbb{C}$}
\begin{tcolorbox}[title=Théorème 18.45, title filled=false, colframe=orange, colback=orange!10!white]
    Soit $f\in \mathcal{C}^1([a, b], \mathbb{C})$ et $M$ un réel tel que $|f'| \leq M$ sur $]a, b[$. Alors
    $$|f(b) - f(a)| \leq M|b - a|$$
\end{tcolorbox}

\noindent Si $f\in C^1([a, b], \mathbb{R})$, alors : 
\begin{align*}
    f(b) - f(a) = \int_{a}^{b} f'(t) \, dt
\end{align*}
D'après l'inégalité triangulaire intégrale :
\begin{align*}
    |f(b) - f(a)| &= \left| \int_{a}^{b} f'(t) \, dt \right| \\
    &\leq \int_{a}^{b} |f'(t)| \, dt \\
    &\leq \int_{a}^{b} M \, dt \\
    &= M|b - a|
\end{align*}


\end{document}