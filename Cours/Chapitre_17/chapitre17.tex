\documentclass[../main.tex]{subfiles}

\begin{document}
\setcounter{chapter}{16}
\chapter{Fractions rationnelles}
\tableofcontents
\clearpage

\setsection{1}
\section{Addition, multiplication et produit par un scalaire}
\begin{tcolorbox}[title=Définition 17.2, title filled=false, colframe=orange, colback=orange!10!white]
    Soit $\frac{P}{Q}$ et $\frac{R}{S}$ deux fractions rationnelles et soit $\lambda \in \mathbb{K}$. On pose
    $$\frac{P}{Q} + \frac{R}{S} = \frac{PS + QR}{QS} \text{, } \frac{P}{Q} \times \frac{R}{S} = \frac{PR}{QS} \text{ et } \lambda \times \frac{P}{Q} = \frac{\lambda P}{Q}.$$
\end{tcolorbox}

\noindent Montrons que l'addition est bien définie. \\
Soit $\frac{P_1}{Q_1} = \frac{P}{Q}$ et $\frac{R}{S}$ dans $\mathbb{K}(X)$. \\
Montrons que : 
\begin{align*}
    \frac{PS + QR}{QS} = \frac{P_1S + Q_1R}{Q_1S}
\end{align*}
On a : 
\begin{align*}
    (PS + QR)Q_1S - (P_1S + Q_1R)QS &= S^2(\underbrace{PQ_1 - P_1Q}_{= 0}) + RS(\underbrace{QQ_1 - Q_1Q}_{= 0}) \\
    &= 0
\end{align*}
On raisonne de la même manière pour $\frac{R}{S} = \frac{R_1}{S_1}$ et ainsi, l'opération est bien définie. 

\setsection{9}
\section{Degré d'une fraction}
\begin{tcolorbox}[title=Définition 17.10, title filled=false, colframe=orange, colback=orange!10!white]
    Soit $F = \frac{P}{Q}$ une fraction. On pose $\deg(F) = -\infty$ si $F = 0$ et $\deg(F) = \deg(P) - \deg(Q)$ sinon. Le degré d'une fraction est donc un élément de $\mathbb{Z} \cup\{ -\infty \}$. 
\end{tcolorbox}

\noindent Si $\frac{P_1}{Q_1} = \frac{P}{Q}$, alors : 
\begin{align*}
    P_1Q &= PQ_1 \\
    \text{donc } \deg(P_1Q) &= \deg(PQ_1) \\
    \text{donc } \deg(P_1) + \deg(Q) &= \deg(P) + \deg(Q_1) \text{ ($\mathbb{K}$ intègre)} \\
    \text{donc } \deg(P_1) - \deg{Q_1} &= \deg(P) - \deg(Q)
\end{align*}

\setsection{12}
\section{Propriété du degré}
\begin{tcolorbox}[title=Théorème 17.13, title filled=false, colframe=orange, colback=orange!10!white]
    Soit $F$ et $G$ deux fractions rationnelles. On a
    $$\deg(F + G) \leq \max(\deg(F), \deg(G)) \text{ et } \deg(F \times G) = \deg(F) + \deg(G).$$
    On retrouve les mêmes propriétés que pour les polynômes. 
\end{tcolorbox}

\noindent Soit $F = \frac{P}{Q}$ et $G = \frac{R}{S}$. 
\begin{itemize}
    \item \begin{align*}
        \deg(F + G) &= \deg(\frac{PS + QR}{QS}) \\
        &= \deg(PS + QR) - \deg(QS) \\
        &\leq \max(\deg(PS), \deg(QR)) - \deg(QS) \\
        &= \max(\deg(PS) - \deg(QS), \deg(QR) - \deg(QS)) \\
        &= \max\left(\deg \left(\frac{P}{Q}\right), \deg \left(\frac{R}{Q}\right)\right) \\
        &= \max(\deg(F), \deg(G))
    \end{align*}

    \item RAS
\end{itemize}

\setsection{18}
\section{Théorème}
\begin{tcolorbox}[title=Théorème 17.19, title filled=false, colframe=orange, colback=orange!10!white]
    Soit $F$ et $G$ deux fractions rationnelles. Si les fonctions rationnelles $\tilde F$ et $\tilde G$ sont égales sur une partie infinie $\mathcal D_F \cap \mathcal D_G$ alors les fractions rationnelles sont égales, i.e. $F = G$. 
\end{tcolorbox}

\noindent On note $F = \frac{P}{Q}$ et $G = \frac{R}{S}$ avec $P \wedge Q = 1$ et $R \wedge S = 1$. \\
On a : 
\begin{align*}
    \forall x \in \mathcal{D} \subset \mathcal{D}_F \cap \mathcal{D}_G, \tilde F(x) = \tilde G(x)
\end{align*}
Soit : 
\begin{align*}
    \forall x \in \mathcal{D}, \tilde{P(x)} \times \tilde{S(x)} = \tilde{R(x)} \times \tilde{Q(x)}
\end{align*}
Comme $\mathcal{D}$ est infini, d'après le théorème de rigidité, $PS = RQ$, donc $F = G$.

\section{Fraction dérivée}
\begin{tcolorbox}[title=Définition 17.20, title filled=false, colframe=orange, colback=orange!10!white]
    Soit $F = \frac{P}{Q} \in \mathbb{K}(X)$. On appelle \textbf{fraction dérivée} de $F$ la fraction notée $F'$ (ou $\frac{dF}{dX}$) définie par
    $$F' = \frac{P'Q - PQ'}{Q^2}.$$
    Le résultat ne dépend pas du représentant de $F$ choisi. On définit également les dérivées successives de $F$ en posant $F^(0) = F$ et pour tout $n \in \mathbb{N}, F^{(n+1)} = (F^{(n)})'$.
\end{tcolorbox}

\noindent On écrit $F = \frac{P}{Q} = \frac{R}{S}$. \\
Montrons que $\frac{P'Q - Q'P}{Q^2} = \frac{R'S - RS'}{S^2}$. \\
Comme $\frac{P}{Q} = \frac{R}{S}$, on a $PS = RQ$. \\
Donc $P'S + S'P = R'Q + Q'R$. \\
Ainsi : 
\begin{align*}
    [P'Q - PQ']S^2 - [R'S - RS']Q^2 &= P'SQ^2 + S'PQ^2 - R'QS^2 - Q'RS^2 \\
    &= QS(P'S - R'Q) + Q^2 RS' - S^2 Q'P \\
    &= QS (Q'R - S'P) + PSQS' - SQRQ' \\
    &= 0
\end{align*}

\setsection{23}
\section{Dérivée logarithmique d'un produit}
\begin{tcolorbox}[title=Théorème 17.24, title filled=false, colframe=orange, colback=orange!10!white]
    Si $F$ est une fraction non nulle qui se facotorise en $F = F_1 \times \ldots \times F_n$ dans $\mathbb{K}(X)$ avec $n \in \mathbb{N}$ alors
    $$\frac{F'}{F} = \frac{F_1'}{F_1} + \ldots + \frac{F_n'}{F_n}.$$
\end{tcolorbox}

\noindent Pour $n = 2$ seulement. \\
\begin{align*}
    F = F_1 \times F_2 \neq 0
\end{align*}
Donc :
\begin{align*}
    F' = F_1'F_2 + F_1F_2'
\end{align*}
Donc :
\begin{align*}
    \frac{F'}{F} = \frac{F_1'F_2}{F_1F_2} + \frac{F_1F_2'}{F_1F_2} = \frac{F_1'}{F_1} + \frac{F_2'}{F_2}
\end{align*}

\section{Partie entière}
\begin{tcolorbox}[title=Théorème 17.25, title filled=false, colframe=orange, colback=orange!10!white]
    Soit $F \in \mathbb{K}(X)$. Il existe un unique polynôme $Q$ tel que $\deg(F - Q) < 0$. Celui-ci est appelé \textbf{partie entière} de $F$, c'est le quotient dans la division euclidienne du numérateur de $F$ par le dénominateur. 
\end{tcolorbox}

\noindent \underline{Existence :} \\
Soit $F = \frac{A}{B}$ avec $A \wedge B = 1$. \\
Soit la division euclidiene de $A$ par $B$ : 
\begin{align*}
    A = BQ + R \text{ avec } \deg(R) < \deg(B)
\end{align*}
Donc : 
\begin{align*}
    F = \frac{A}{B} = \frac{BQ + R}{B} = Q + \frac{R}{B}
\end{align*}
Donc : 
\begin{align*}
    \deg(F - Q) = \deg\left(\frac{R}{B}\right) = \deg(R) - \deg(B) < 0
\end{align*} \\

\noindent \underline{Unicité :} \\
On suppose que : 
\begin{align*}
    F = Q + G = Q_1 + G_1 \text{ avec } (Q_1, G_1) \in \mathbb{K}[X]^2 \text{ et } \deg(G), \deg(G_1) < 0
\end{align*}
Donc :
\begin{align*}
    Q - Q_1 &= G_1 - G \\
    \text{donc } \deg(Q - Q_1) &= \deg(G_1 - G) \\
    &\leq \max(\deg(G_1), \deg(G)) \\
    &< 0
\end{align*}
Or $Q - Q_1 \in \mathbb{K}[X]$, donc $Q = Q_1$. 

\setsection{30}
\section{Existence d'une décomposition}
\begin{tcolorbox}[title=Théorème 17.31, title filled=false, colframe=orange, colback=orange!10!white]
    Si $T$ et $S$ sont deux polynômes premiers entre eux et si $\deg \left( \frac{A}{TS} \right) < 0$, alors il existe deux polynômes $U$ et $V$ tels que
    $$\frac{A}{TS} = \frac{U}{T} + \frac{V}{S} \text{, avec } \deg(U) < \deg(T) \text{ et } \deg(V) < \deg(S).$$
\end{tcolorbox}

\noindent Comme $T \wedge S = 1$, d'après le théormème de Bézout, on écrit : 
\begin{align*}
    CT + DS = 1
\end{align*}
Donc : 
\begin{align*}
    ACT + DSA = A
\end{align*}
Donc : 
\begin{align*}
    \frac{A}{TS} &= \frac{ACT + DSA}{TS} \\
    &= \frac{DA}{T} + \frac{AC}{S}
\end{align*}
On écrit la division euclidienne de $DA$ par $T$ et de $AC$ par $S$ :
\begin{align*}
    DA &= TQ + U \text{ avec } \deg(U) < \deg(T) \\
    AC &= SH + V \text{ avec } \deg(V) < \deg(S)
\end{align*}
Donc :
\begin{align*}
    \frac{A}{TS} = \frac{U}{T} + \frac{V}{S} + Q + H
\end{align*}
Ainsi : 
\begin{align*}
    \deg (Q + H) &= \deg \left( \frac{A}{TS} - \frac{U}{T} - \frac{V}{S} \right) \\
    &\leq \max(\ldots, \ldots, \ldots) \\
    &< 0
\end{align*}
Donc $Q + H = 0$. 

\section{Théorème}
\begin{tcolorbox}[title=Théorème 17.33, title filled=false, colframe=orange, colback=orange!10!white]
    Si $T$ est un polynôme irréductible unitaire et si $\deg \left( \frac{A}{T^n} \right) < 0$ (avec $n \geq 1$), alors il existe des polynômes $V_1, \ldots, V_n$ tels que
    $$\frac{A}{T^n} = \sum_{k=1}^{n} \frac{V_k}{T^k} \text{, avec } \deg(V_k) < \deg(T).$$
    C'est une \textbf{décomposition en éléments simples}.
\end{tcolorbox}

\noindent Par récurrence sur $n$. \\
\begin{itemize}
    \item Pour $n = 1$, RAF. 
    \item On suppose le résultat vrai pour $n \geq 1$ fixé. \\
    On écrit la division euclidienne de $A$ par $T$ :
    \begin{align*}
        A = BT + V_{n+1} \text{ avec } \deg(V_{n+1}) < \deg(T)
    \end{align*}
    Ainsi : 
    \begin{align*}
        \frac{A}{T^{n+1}} &= \frac{BT + V_{n+1}}{T^{n+1}} \\
        &= \frac{B}{T^n} + \frac{V_{n+1}}{T^{n+1}} \\
        &= \sum_{k=1} \frac{V_k}{T^k} + \frac{V_{n+1}}{T^{n+1}} \text{ (Hypothèse de récurrence)} 
    \end{align*}
\end{itemize}

\setsection{37}
\section{Cas d'un pôle simple}
\begin{tcolorbox}[title=Propostion 17.38, title filled=false, colframe=lightblue, colback=lightblue!10!white]
    Si $a$ est un pôle simple de $F = \frac{A}{B}$, alors la partie polaire de $F$ relative à $a$ est
    $$P_F(a) = \frac{c}{X - a} \text{ avec } c = \frac{A(a)}{B'(a)} = \frac{A(a)}{Q(a)} \text{ où } B = (X - a)Q.$$
\end{tcolorbox}

\noindent D'après le théorème d'existence de la DES : 
\begin{align*}
    \frac{A}{B} = F = E + \frac{c}{X - a} + G
\end{align*}
Donc : 
\begin{align*}
    c &= \frac{(X - a)A}{B} - (X - a)E - (X - a)G \\
    &= \frac{A}{Q} - (X - a)E - (X - a)G \\
\end{align*}
Donc $c = \frac{A(a)}{Q(a)}$. \\
Si $B = (X - a)Q$, alors $B'(a) = Q(a)$.

\section{Exemple}
\begin{tcolorbox}[title=Exemple 17.39, title filled=false, colframe=darkgreen, colback=darkgreen!10!white]
    Décomposer en éléments simples dan $\mathbb{C}(X)$ la fraction raitonnelle $F = \frac{1}{X^n - 1}$ avec $n \geq 1$. 
\end{tcolorbox}

\begin{itemize}
    \item $\deg F = -n < 0$. \\
    \item $F$ possède $n$ pôles simples. $e^{\frac{2ik\pi}{n}} = \omega_k$. 
    \item D'après le théorème de DES : 
    \begin{align*}
        F = \sum_{k=0}^{n-1} \frac{c_k}{X - \omega_k}
    \end{align*}
    Or, pour tout $k \in \llbracket 0, n-1 \rrbracket$, $c_k = \frac{1}{nw_k^{n-1}} = \frac{\omega_k}{n}$. \\
    \begin{align*}
        F = \frac{1}{n} \sum_{k=0}^{n-1} \frac{\omega_k}{X - \omega_k}
    \end{align*}
\end{itemize}

\section{Cas d'un pôle double}
\begin{tcolorbox}[title=Propostion 17.40, title filled=false, colframe=lightblue, colback=lightblue!10!white]
    Si $a$ est un pôle double de $F = \frac{A}{B}$, alors la partie polaire de $F$ relative à $a$ est
    $$P_F(a) = \frac{\alpha}{X - a} + \frac{\beta}{(X - a)^2} \text{ avec } \beta = H(a) \text{ et } \alpha = H'(a) \text{ en posant } H = (X - a)^2 F.$$
\end{tcolorbox}

\noindent On a (notations 17.38) : 
\begin{align*}
    F &= E + \frac{\alpha}{X - a} + \frac{\beta}{(X - a)^2} + G \\
    \beta + (X - a) \alpha &= \underbrace{(X - a)^2 F}_{:= H} - (X - a)^2 E - (X - a)^2 G \\
\end{align*}
En évaluant en $a$ : $\beta = H(a)$. \\
On dérive et on évalue en $a$ : $\alpha = H'(a)$.

\setsection{41}
\section{Exemple}
\begin{tcolorbox}[title=Exemple 17.42, title filled=false, colframe=darkgreen, colback=darkgreen!10!white]
    Décomposer $F = \frac{X^6}{(X - 1)^2(X^3 + 1)}$ en éléments simples dans $\mathbb{C}(X)$. 
\end{tcolorbox}

\begin{itemize}
    \item $\deg F = 1 \geq 0$
    \item \begin{align*}
        X^6 = (X-1)^2(X^3 + 1)(X + 2) + R \text{ avec } \deg R < 5
    \end{align*}
    \item D'après le théorème DES : 
    \begin{align*}
        F &= \frac{X^6}{(X - 1)^2(X + 1)(X + j)(X + j^2)} \\
        &= X + 2 + \frac{a}{X - 1} + \frac{b}{(X - 1)^2} + \frac{c}{X + 1} + \frac{d}{X + j} + \frac{e}{x + j^2} \\
        c &= (x+1)\tilde F (-1) = \frac{1}{4} \\
        d &= (x+j)\tilde F (-j) \\
        &= \frac{1}{(j+1)^2(1-j)(-j+j^2)} \\
        &= \frac{1}{(1+j)(1-j^2)(j-1)j} \\
        &= \frac{-1}{(1-j^2)^2j} \\
        &= \frac{-1}{j(-3j^2)} \\
        &= \frac{1}{3} \\
        e &= (x+j^2)\tilde F (-j^2) = \frac{1}{3} \\
        H &= (X-1)^2 F = \frac{X^6}{X^3 + 1} \\
        b &= H(1) = \frac{1}{2} \\
        a = H'(1) &= \frac{9}{4}
    \end{align*}
\end{itemize}

\setsection{43}
\section{Parties polaires conjuguées d'une fraction réelle}
\begin{tcolorbox}[title=Propostion 17.44, title filled=false, colframe=lightblue, colback=lightblue!10!white]
    Si $F$ est à coefficients réels, alors les parties polaires relatives aux pôles conjugués sont conjuguées. 
\end{tcolorbox}

\noindent Soit $F \in \mathbb{R}(X) \subset \mathbb{C}(X)$. \\
On écrit $F = \frac{A}{B}$ avec $A, B \in \mathbb{R}(X)^2$. \\
Soit $r$ un pôle de multiplicité $m$. \\
Comme $F \in \mathbb{R}(X)$, $\overline{r}$ est un pôle de multiplicité $m$. On suppose que $r \neq \overline{r}$ \\
D'après le théorème de DES, on écrit : 
\begin{align*}
    F = E + P_F(r) + G \text{ avec } (E, r) \in \mathbb{R}(X)^2, G \in \mathbb{C}(X)
\end{align*}
$r$ n'est pas un pôle de $G$ ($\overline{r}$ oui). \\
Ainsi : 
\begin{align*}
    F &= \overline{F} \\
    &= \overline{E + P_F(r) + G} \\
    &= \overline{E} + P_F(\overline{r}) + \overline{G} \\
    &= E + \overline{P_F(r)} + \overline{G}
\end{align*}
Or $r$ n'est pas un pôle de $\overline{P_F(r)}$ mais $\overline{r}$ est un pôle de $\overline{P_F(r)}$. \\
De la même manière, comme $r$ n'est pas un pôle de $G$, $\overline{r}$ n'est pas un pôle de $\overline{G}$. \\
Donc $P_F(\overline{r}) = \overline{P_F(r)}$.


\end{document}