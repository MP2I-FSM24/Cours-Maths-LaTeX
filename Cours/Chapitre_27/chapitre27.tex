\documentclass[../main.tex]{subfiles}

\begin{document}
\setcounter{chapter}{26}
\chapter{Séries numériques}
\tableofcontents
\clearpage

\setsection{5}
\section{Série géométrique}
\begin{tcolorbox}[title=Théorème 27.6, title filled=false, colframe=orange, colback=orange!10!white]
    Soit $a\in \mathbb{C}$. La série $\sum a^n$ converge si et seulement si $|a|<1$. Dans ce cas : 
    \begin{align*}
        \sum_{n=0}^{+\infty} a^n = \frac{1}{1-a}
    \end{align*}
\end{tcolorbox}

\noindent Soit $n\in \mathbb{N}$. 
\begin{align*}
    S_n = \sum_{k=0}^{n} a^k &= \frac{1-a^{n+1}}{1-a} \text{ ($a\neq 1$)} \\
    &\underset{n \to +\infty}{\longrightarrow} \frac{1}{1-a} \text{ ($|a|<1$)}
\end{align*}
La série converge et $\sum\limits_{n\geq 0} a^n = \frac{1}{1-a}$. 


\end{document}