\documentclass[../main.tex]{subfiles}

\begin{document}
\setcounter{chapter}{11}
\chapter{Arithmétique}
\tableofcontents
\clearpage

\section{Propriété fondamentale de $\mathbb{Z}$}
\begin{tcolorbox}[title=Théorème 12.1, title filled=false, colframe=orange, colback=orange!10!white]
    Toute partie non vide et minorée de $\mathbb{Z}$ admet un plus petit élément. 
\end{tcolorbox}

Soit $A$ une partie non vide et minorée de $\mathbb{Z}$. \\
On note $\mathcal{M}$ l'ensemble des minorants de $A$. \\
Par hypothèse, $\mathcal{M} \neq \emptyset$. \\
Supposons par l'absurde que : 
\begin{align*}
    \forall a \in \mathbb{Z}, a \in \mathcal{M} \Rightarrow a + 1 \in \mathcal{M}
\end{align*}
D'après le principe de récurrence, si $a_0 \in \mathcal{M}$ est fixé : 
$$\forall n \geq a_0, n \in \mathcal{M}$$
En particulier, pour $n \in A$ ($A \neq \emptyset$) on a : 
\begin{align*}
    n \geq a_0 \text{ ($a_0$ est un minorant)}
\end{align*}
Donc $n \in \mathcal{M}$. \\
Donc $n + 1 \in \mathcal{M}$. \\
Donc $n + 1$ est un minorant de $A$. \\
Donc $n + 1 \leq n$. \\
Absurde. \\
Ainsi, on choisit $a \in \mathbb{Z}$ avec $a \in \mathcal{M}$ et $a + 1 \not \in \mathcal{M}$. \\
On choisit donc $n \in A$ tel que : 
$$a \leq n < a + 1$$
Donc $n = a \in A$. \\
Donc $a = \min(A)$.

\setcounter{section}{3}
\section{Division euclidienne}
\begin{tcolorbox}[title=Théorème 12.4, title filled=false, colframe=orange, colback=orange!10!white]
    Soit $(a,b) \in \mathbb{Z} \times \mathbb{Z}^*$. Il existe un unique coupe $(q,r) \in \mathbb{Z} \times \mathbb{N}$ tel que :
    $$a = bq + r$$
    avec $0 \leq r < |b|$. Cette égalité est appelée \textbf{division euclidienne de $a$ par $b$}, l'entier $q$ est alors appelé \textbf{quotient} et l'entier $r$ le \textbf{reste}, tandis que $a$ porte le nom de dividende et $b$ celui de diviseur. 
\end{tcolorbox}
\underline{Existence :} \\
On suppose dans un premier temps que $b > 0$. \\
Soit $a \in \mathbb{Z}$. \\
On note $A = \{ n \in \mathbb{Z}, bn \leq a \}$. \\
$A$ est un sous-ensemble non vide de $\mathbb{Z}$ et majoré. \\
Il admet donc un plus grand élément, noté $q$. 
On a donc $q \in A$ et $q + 1 \not \in A$. 
\begin{align*}
    &bq \leq a < b(q + 1) \\
    \text{donc } &0 \leq a - bq < b
\end{align*}
On pose alors $r = a - bq$. L'exsitence est alors prouvée pour $b > 0$. \\
Si $b < 0$, alors $-b > 0$ et on choisit $(q,r) \in \mathbb{Z}^2$ tel que :
\begin{align*}
    a = -b \times q + r \text{ avec } 0 \leq r < -b
\end{align*}
Le couple $(-q, r)$ convient. \\ \\

\underline{Unicité :} \\
On suppose $a = bq + r = bq' + r'$ avec $0 \leq r, ' < |b|$. \\
Donc $b(q-q') = r' - r$. \\
Donc $\underbrace{|b|}_{> 0} \times |q - q'| = |r' - r| < \underbrace{|b|}_{> 0}$. \\
Donc $|q - q'| < 1$. \\
Donc $q = q'$. \\
Puis $r = r'$. 

\setcounter{section}{8}
\section{Divisibilité et multiple}
\begin{tcolorbox}[title=Propostion 12.9, title filled=false, colframe=lightblue, colback=lightblue!10!white]
    Soit $a$ et $b$ deux entiers. Alors $a$ est divisble par $b$ si et seulement si $a$ est un multiple de $b$.
\end{tcolorbox}

$\boxed{\Rightarrow}$ \\
Si $b|a$, alors : 
\begin{align*}
    a &= bq + 0 \\
    &= bq \\
    &\in b \mathbb{Z}
\end{align*} \\ \\

$\boxed{\Leftarrow}$ \\
Si $a \in b \mathbb{Z}$, $a = b \times n$ = $b \times n + 0$. \\
Par unicité de la division euclidienne, $b|a$.

\section{Divisibilité et normes}

\begin{tcolorbox}[title=Propostion 12.10, title filled=false, colframe=lightblue, colback=lightblue!10!white]
    Soit $a$ et $b$ deux entiers avec $a \neq 0$ et $b|a$. Alors $|b| \leq |a|$. 
\end{tcolorbox}

Si $b|a$, alors $a = b \times n$ avec $n \neq 0$ var $a \neq 0$. 
Donc : 
\begin{align*}
    |a| &= |b| \times |n| \\
    &\geq |b| \times 1
\end{align*}

\section{Entiers associés}
\begin{tcolorbox}[title=Propostion 12.11, title filled=false, colframe=lightblue, colback=lightblue!10!white]
    Soit $a$ et $b$ deux entiers. Alors 
    $$a \mathbb{Z} = b \mathbb{Z} \Leftrightarrow a = \pm b$$
    On dit alors que $a$ et $b$ sont associés. 
\end{tcolorbox}

$\boxed{\Leftarrow}$ \\
Si $a = \pm b$, alors $a \mathbb{Z} = b \mathbb{Z}$. \\ \\

$\boxed{\Rightarrow}$ \\
Si $a = 0$ et $a \mathbb{Z} = b \mathbb{Z}$, alors $b = 0$. \\
Si $a \neq 0$ et $a \mathbb{Z} = b \mathbb{Z}$, alors $b \neq 0$ et d'après $\text{(12.0)}$ : 
$$|a| \leq |b| \text{ et } |b| \leq |a|$$
Donc $|a| = |b|$

\setcounter{section}{13}
\section{Intégrité de la divisibilité}
\begin{tcolorbox}[title=Propostion 12.14, title filled=false, colframe=lightblue, colback=lightblue!10!white]
    Soit $a$, $b$ et $c$ trois entiers, avec $c \neq 0$. Si $nb|na$, alors $n|a$.
\end{tcolorbox}

Si $cb|ca$, alors $ca = ncb$. \\
Or $c$ est régulier dans $\mathbb{Z}$ donc :
$$a = nb$$
Donc $b|a$. 

\setcounter{section}{19}
\section{Cas d'une divisibilité}
\begin{tcolorbox}[title=Lemme 12.20, title filled=false, colframe=orange, colback=orange!10!white]
    Si $a|b$, alors 
    $$\mathcal{D}_{a,b} = \mathcal{D}_a$$
\end{tcolorbox}

Si $a|b$, si $c|a$, alors $c|b$. \\
Donc $\mathcal{D}_b \supset \mathcal{D}_a$. \\
Ainsi, $\mathcal{D}_a \cap \mathcal{D}_b = \mathcal{D}_a$


\section{Préparation à l'algorithme d'Euclide}
\begin{tcolorbox}[title=Lemme 12.21, title filled=false, colframe=orange, colback=orange!10!white]
    Soit $a$, $b$ et $q$ trois entiers, alors
    $$\mathcal{D}_{a,b} = \mathcal{D}_{a - bq, b}$$
\end{tcolorbox}

$\boxed{\subset}$ \\
Soit $n \in \mathcal{D}_{a,b}$, alors : 
\begin{align*}
    n|a \text{ et } n|b \\
    \text{donc } n|a - bq \\
    \text{donc } n \in \mathcal{D}_{a - bq, b}
\end{align*} \\

$\boxed{\supset}$ \\
Soit $n \in \mathcal{D}_{a - bq, b}$
\begin{align*}
    n|a - bq \text{ et } n|b \\
    \text{donc } n|a - bq + bq \\
    \text{soit n|a} \\
    \text{donc } n \in \mathcal{D}_{a,b}
\end{align*}

\setcounter{section}{22}
\section{Algorithme d'Euclide étendu ou théorème de Bézout}
\begin{tcolorbox}[title=Lemme 12.23, title filled=false, colframe=orange, colback=orange!10!white]
    Soit $a$ et $b$ deux entiers. Soit $r$ le dernier reste non nul dans l'algorithme d'Euclide appliqué à $a$ et $b$. Il existe deux entiers $u$ et $v$ tels que
    $$au + bv = r$$
\end{tcolorbox}

On utilise les notations du lemme $\text{(12.22)}$. \\
On démontre par récurrence double que : 
$$\forall n, "\exists (u_n, v_n) \in \mathbb{Z}^2, au_n + bv_n = r_n"$$

\underline{Initialisation :} \\
Pour $n = 0$ il s'agit de la division euxlidienne de $a$ par $b$ ($u_0 = $ et $v_0 = -q$). \\
Pour $n = 1$ : 
\begin{align*}
    a &= bq + r \\
    b &= r \times q_1 + r_1 \\
    \text{donc } r &= b - rq_1 \\
    &= b - q_1(a - bq) \\
    &= -q_1a + b(1 + q_1q)
\end{align*}

\underline{Hérédité :} \\
On suppose le résultat vrai aux rangs $n$ et $n + 1$. \\
\begin{align*}
    a_n &= b_nq_n + r_n \\
    b_n &= r_nq_{n+1} + r_{n+1} \\
    r_n &= r_{n+1}q_{n+2} + r_{n+2} \\
\end{align*}
Donc : 
\begin{align*}
    r_{n+2} &= r_n - r_{n+1}q_{n+2} \\
    &=au_n + bv_n - (au_{n+1} + bv_{n+1})q_{n+2} \\
    &= a \underbrace{(u_n - u_{n+1}q_{n+2})}_{\in \mathbb{Z}} + b \underbrace{(v_n - v_{n+1}q_{n+2})}_{\in \mathbb{Z}}
\end{align*}
On utilise le principe de récurrence avec la dernière étape de l'algorithme. 

\section{Application basique}
\begin{tcolorbox}[title=Exemple 12.24, title filled=false, colframe=darkgreen, colback=darkgreen!10!white]
    Appliquer l'algorithme d'Euclide aux entiers $121$ et $26$. 
\end{tcolorbox}

\begin{align*}
    121 &= 26 \times 4 + 17 \\
    26 &= 17 \times 1 + 9 \\
    17 &= 9 \times 1 + 8 \\
    9 &= 8 \times 1 + 1 \\
    8 &= 1 \times 8 + 0
\end{align*}
On remonte l'algorithme : 
\begin{align*}
    1 &= 9 - 8 \\
    &= 9 - (17 - 9) \\
    &= 2 \times 9 - 17 \\
    &= 2 \times (26 - 17) - 17 \\
    &= 2 \times 26 - 3 \times 17 \\
    &= 2 \times 26 - 3 \times (121 - 4 \times 26) \\
    &= 14 \times 26 - 3 \times 121
\end{align*}

\setcounter{section}{25}
\section{Théorème de Bézout}
\begin{tcolorbox}[title=Théorème 12.26, title filled=false, colframe=orange, colback=orange!10!white]
    Soit $a$ et $b$ deux entiers. Alors $a$ et $b$ sont premiers entre eux si et seulement si il existe $(u,v) \in \mathbb{Z}^2$ tel que
    $$au + bv = 1$$
\end{tcolorbox}

$\boxed{\Rightarrow}$ \\
On suppose $a$ et $b$ premiers entre eux. \\
Donc $\mathcal{D}_{a,b} = \{\pm 1\}$. \\
Soit $r$ le dernier reste non nul dans l'algorithme d'Euclide, \\
$$\mathcal{D}_r = \mathcal{D}_{a,b} = \{\pm 1\}$$
Donc $r = \pm 1$. \\
D'après le théorème de Bézout, il existe deux entiers $u$ et $v$ tels que : 
$$au + bv = 1$$ \\

$\boxed{\Leftarrow}$ \\
Réciproquement, si $au + bv = 1$, alors pour tout $d \in \mathcal{D}_{a,b}$ $d|au + bv$ donc $d|1$ donc $d = \pm 1$. \\
Donc $\mathcal{D}_{a,b} = \{\pm 1\}$. 

\setcounter{section}{27}
\section{Proposition}
\begin{tcolorbox}[title=Propostion 12.28, title filled=false, colframe=lightblue, colback=lightblue!10!white]
    Si $a$ est premier avec $b$ et $c$, alors $a$ est premier avec $bc$.
\end{tcolorbox}
D'après le théorème de Bézout, on écrit : 
\begin{align*}
    au_1 + bv_1 &= 1 \\
    au_2 + cv_2 &= 1
\end{align*}
avec $(u_1, u_2, v_1, v_2) \in \mathbb{Z}^4$. \\
Donc :
\begin{align*}
    1 &= (au_1 + bv_1)(au_2 + cv_2) \\
    &= a \underbrace{(au_1u_2 + bv_1u_2 + cu_1v_2)}_{\in \mathbb{Z}} + \underbrace{v_1v_2}_{\in \mathbb{Z}} bc
\end{align*}
Donc $a$ et $bc$ sont premiers entre eux d'après le théorème de Bézout.

\section{Proposition}
\begin{tcolorbox}[title=Propostion 12.29, title filled=false, colframe=lightblue, colback=lightblue!10!white]
    Si $a$ est premier avec $b$, que $a|c$ et $b|c$, alors $ab|c$.
\end{tcolorbox}
D'après le théorème de Bézout :
$$au + bv = 1 \text{, } (u,v) \in \mathbb{Z}^2$$
Donc : 
$$auc + bvc = c$$
Or $a|c$ et $b|c$, donc : \\
$$c = ka \text{ et } c = pb$$
Donc :
$$ab \underbrace{[pu + vk]}_{\in \mathbb{Z}} = c$$
Donc $ab|c$.

\section{Théorème de Gauss}
\begin{tcolorbox}[title=Théorème 12.30, title filled=false, colframe=orange, colback=orange!10!white]
    Si $a|bc$ et que $a$ est premier avec $b$, alors $a|c$.
\end{tcolorbox}
D'après le théorème de Bézout : 
$$au + bv = 1 \text{ avec } (u,v) \in \mathbb{Z}^2$$
Donc $auc + bvc = c$. \\
Or $a|bc$ donc $a|auc + bvc$. \\
Soit $a|c$.

\section{Equation de Bézout}
\begin{tcolorbox}[title=Exemple 12.31, title filled=false, colframe=darkgreen, colback=darkgreen!10!white]
    Résoudre l'équation d'inconnue $(x, y) \in \mathbb{Z}^2, 3x - 2y = 7$. 
\end{tcolorbox}
On remarque que $3$ et $2$ sont premiers entre eux. \\
\begin{align*}
    3 - 2 &= 1 \\
    \text{donc } 3 \times 7 - 2 \times 7 &= 7 \\
    \text{donc } (7, 7) &\in \mathcal{S}
\end{align*}
On note $(x_0, y_0)$ cette solution. \\
Soit $(x, y) \in \mathcal{S}$. \\
Donc : 
\begin{align*}
    7 &= 3x - 2y \\
    7 &= 3x_0 - 2y_0 \\
    \text{donc } 3(x - x_0) &= 2(y - y_0) \\
\end{align*}
Or $3|3(x - x_0)$ et $3$ premier avec $2$. \\
Donc $3|y - y_0$. \\
Donc $y - y_0 = 3k$, avec $k \in \mathbb{Z}$. (Théorème de Gauss) \\
De la même manière, $x - x_0 = 2l$, avec $l \in \mathbb{Z}$. (Théorème de Gauss) \\ \\

\noindent Réciproquement, soit $x = x_0 + 2l$ et $y = y_0 + 3k$. \\
\begin{align*}
    (x, y) \in \mathcal{S} &\Leftrightarrow 7 = 3x - 2y = 3x_0 - 2y_0 + 6l - 6k \\
    &\Leftrightarrow 6l - 6k = 0 \\
    &\Leftrightarrow k = l
\end{align*}
Donc $\mathcal{S} = \{(x_0 + 2k, y_0 + 3k), k \in \mathbb{Z}\}$

\section{Proposition}
\begin{tcolorbox}[title=Propostion 12.32, title filled=false, colframe=lightblue, colback=lightblue!10!white]
    Si $ar \equiv br \mod n$ et si $r$ et $n$ sont premiers entre eux, alors $a \equiv b \mod n$.
\end{tcolorbox}
Si $ar \equiv br \mod n$, alors $n|r(a-b)$. \\
Donc $n|a-b$ ($n$ premier avec $r$ et théorème de Gauss). \\
Donc $a \equiv b \mod n$.

\setcounter{section}{36}
\section{Lien avec les idéaux}
\begin{tcolorbox}[title=Propostion 12.37, title filled=false, colframe=lightblue, colback=lightblue!10!white]
    Soit $a$ et $b$ deux entiers, alors $d$ est le $pgcd$ de $a$ et $b$ si et seulement si $a \mathbb{Z} + b \mathbb{Z} = d \mathbb{Z}$.
\end{tcolorbox}

Soit $(a,b) \in \mathbb{Z}^2$. $a \mathbb{Z} \text{ et } b \mathbb{Z}$ dont des idéaux de $\mathbb{Z}$. \\
Donc $a \mathbb{Z} + b \mathbb{Z}$ est un idéal de $\mathbb{Z}$, donc en particulier un sous-groupe de $\mathbb{Z}$. \\
On choisit donc $d \geq 0$ tel que $a \mathbb{Z} + b \mathbb{Z} = d \mathbb{Z}$. \\
Montrons que $d = pgcd(a,b) = a \wedge b$. \\
D'une part : 
\begin{align*}
    d &\in d \mathbb{Z} && \text{donc } d = au + bv \text{ (avec $(u, v) \in \mathbb{Z}^2$} \\
    &\in a \mathbb{Z} + b \mathbb{Z} \\
    \text{or } a \wedge b | a &\text{ et } a \wedge b | b && \text{donc } a \wedge b | au + bv \\
    & && \text{soit } a \wedge b | d
\end{align*}
D'autre part, $a \wedge b$ est le dernier reste non nul de l'algorithme d'Euclide, donc $(\text{12.23})$ :
\begin{align*}
    a \wedge b &= au + bv \text{ (avec $(u, v) \in \mathbb{Z}^2$)} \\
    &\in a \mathbb{Z} + b \mathbb{Z} \\
    &\in d \mathbb{Z} \\
\end{align*}
Donc $d | a \wedge b$. \\
Ainsi, $d$ et $a \wedge b$ sont positifs et associés, donc égaux.

\section{Préparation au calcul pratique d'un $pgcd$}
\begin{tcolorbox}[title=Lemme 12.38, title filled=false, colframe=orange, colback=orange!10!white]
    Si $a$ et $b$ sont tous les deux non nuls, alors pour tout $q \in \mathbb{Z}, pgcd(a,b) = pgcd(a - bq, b)$.
\end{tcolorbox}
\begin{align*}
    \mathcal{D}_{pgcd(a,b)} &= \mathcal{D}_{a,b} \\
    &\underset{\text{(12.21)}}{=} \mathcal{D}_{a - bq, b} \\
    &= \mathcal{D}_{pgcd(a - bq, b)}
\end{align*}
Les deux $pgcd$ sont associés, donc égaux car positifs. 

\section{Caractérisation du $pgcd$}
\begin{tcolorbox}[title=Propostion 12.39, title filled=false, colframe=lightblue, colback=lightblue!10!white]
    Soit $a$ et $b$ deux entiers et $d \in \mathbb{N}$. Alors $d = pgcd(a,b)$ si et seulement si il existe $(u, v) \in \mathbb{Z}^2$ avec $u$ et $v$ premiers entre eux, tels que $a = du$ et $b = dv$. 
\end{tcolorbox}

$\boxed{\Rightarrow}$ \\
On suppose que $d = a \wedge b$. \\
Donc $d|a$ et $d|b$. \\
On écrit donc $a = du$ et $b = dv$ avec $(u, v) \in \mathbb{Z}^2$. \\
Notons $n = u \wedge v$. On écrit $u = n \times u'$ et $v = n \times v'$ avec $(u', v') \in \mathbb{Z}^2$.  \\
Donc $a = d \times n \times u'$ et $b = d \times n \times v'$. \\
Donc $dn \in \mathcal{D}_{a,b} = \mathcal{D}_d$. \\
Donc $dn | d$. \\
Donc $n = 1$. \\

$\boxed{\Leftarrow}$ \\
On suppose que $a = du$ et $b = dv$ avec $u \wedge v = 1$. \\
D'après le théorème de Bézout : 
$$uu' + vv' = 1 \text{ (avec $(u', v') \in \mathbb{Z}^2$)}$$
Donc $duu' + dvv' = d$. \\
Soit $au' + bv' = d$. \\
Donc $d \in a \mathbb{Z} + b \mathbb{Z} = (a \wedge b) \mathbb{Z}$. \\
Donc $a \wedge b | d$. \\
Par ailleurs, $d \in \mathcal{D}_{a, b} = \mathcal{D}_{a \wedge b}$. \\
Donc $d | a \wedge b$. \\
Ainsi, $a \wedge b$ et $d$ sont associés (et positifs) donc égaux. 

\section{Propriétés du $pgcd$}
\begin{tcolorbox}[title=Propostion 12.40, title filled=false, colframe=lightblue, colback=lightblue!10!white]
    Soit $a$ et $b$ deux entiers tous deux non nuls. 
    \begin{enumerate}
        \item pour tout $n \in \mathbb{Z}$, si $n|a$ et $n|b$, alors $n|pgcd(a,b)$;
        \item pour tout $k \in \mathbb{N}^*, pgcd(ka, kb) = kpgcd(a,b)$;
        \item pour tout $n \in \mathbb{N}, pgcd(a^n, b^n) = pgcd(a,b)^n$;
        \item si $a$ et $c$ sont premiers entre eux, alors $pgcd(a,bc) = pgcd(a,b)$.
    \end{enumerate}
\end{tcolorbox}

\begin{enumerate}
    \item RAF (définition)
    \item Soit $k \in \mathbb{N}^*$. On écrit $\text{(12.39)}$ : 
    \begin{align*}
        a &= (a \wedge b)u \\
        b &= (a \wedge b)v \text{ (avec $u \wedge v = 1$)}
    \end{align*}
    Donc : 
    \begin{align*}
        ka &= \left[ k(a \wedge b) \right]u \\
        kb &= \left[ k(a \wedge b) \right]v
    \end{align*}
    Donc $(12.39)$ : 
    $$pgcd(ka, kb) = k(a \wedge b)$$

    \item Avec une partie des notations de $2.$ : 
    \begin{align*}
        a^n &= (a \wedge b)^n u^n \\
        b^n &= (a \wedge b)^n v^n
    \end{align*}
    Avec $(u^n) \wedge (v^n) = 1$. \\
    Donc $(12.39)$ :
    $$pgcd(a^n, b^n) = (a \wedge b)^n$$

    \item \begin{align*}
        a &= (a \wedge b)u \\
        b &= (a \wedge b)v \text{ (avec $u \wedge v = 1$)}
    \end{align*}
    Donc
    $$bc = (a \wedge b) \times vc$$
    Or, puisque $a \wedge c = 1$ et que $u | a$, alors : 
    $$u \wedge c = 1$$
    Donc $\text{(12.28)}$ :
    $$u \wedge (vc) = 1$$
    Donc $\text{(12.39)}$ :
    $$pgcd(a, bc) = a \wedge b$$
\end{enumerate}

\setcounter{section}{43}
\section{Définition du PPCM}

\begin{tcolorbox}[title=Propostion 12.44, title filled=false, colframe=lightblue, colback=lightblue!10!white]
    Soit $a$ et $b$ deux entiers non nuls. On appelle \textbf{PPCM} (plus petit commun multiple) l'unique entier $m \in \mathbb{N}$ tel que
    $$(a \mathbb{Z}) \cap (b \mathbb{Z}) = m \mathbb{Z}.$$
    Cet entier est noté $ppcm(a,b)$ ou encore $a \vee b$.
\end{tcolorbox}

$a \mathbb{Z}$ et $b \mathbb{Z}$ ont des idéaux de $\mathbb{Z}$. \\
Donc $a \mathbb{Z} \cap b \mathbb{Z}$ est un idéal de $\mathbb{Z}$, donc un sous-groupe de $\mathbb{Z}$. \\
Donc il existe un unique entier $m \in \mathbb{N}$ tel que :
$$a \mathbb{Z} \cap b \mathbb{Z} = m \mathbb{Z}$$
Comme $a \neq 0$ et $b \neq 0$, alors $m \neq 0$. 

\section{Caractérisation du $ppcm$}
\begin{tcolorbox}[title=Propostion 12.45, title filled=false, colframe=lightblue, colback=lightblue!10!white]
    Soit $a$ et $b$ deux entiers, et $m \in \mathbb{N}$. Alors $m = ppcm(a,b)$ si et seulement si il existe $(u, v) \in \mathbb{Z}^2$, premiers entre eux tels que $m = au = bv$. 
\end{tcolorbox}

$\boxed{\Rightarrow}$ \\
On suppose que $m = a \vee b$. \\
Donc $m \in a \mathbb{Z} \cap b \mathbb{Z}$. \\
Donc $m = au = bv$. \\
On note $d = pgcd(u,v)$. \\
On écrit donc : 
\begin{align*}
    u &= da' \\
    v &= db'
\end{align*}
Donc : 
$$ada' = bdb'$$
Donc : 
$$aa' = bb' = m'$$
Donc : 
\begin{align*}
    m' &\in a \mathbb{Z} \cap b \mathbb{Z} \\
    &\in m \mathbb{Z}
\end{align*}
Donc : 
$$dm' = m | m'$$
Donc : 
$$d = 1$$

$\boxed{\Leftarrow}$ \\
On suppose que $m = au = bv$ avec $pgcd(u, v) = 1$. \\
D'une part : 
$$m \in a \mathbb{Z} \cap b \mathbb{Z} = ppcm(a, b) \mathbb{Z}$$
Donc : 
$$ppcm(a, b) | m$$
D'autre part, d'après le théorème de Bézout :
$$uu' + vv' = 1 \text{ avec $(u', v') \in \mathbb{Z}^2$}$$
Donc : 
$$uu'\underbrace{ppcm(a, b)}_{ka} + vv'\underbrace{ppcm(a, b)}_{qb} = ppcm(a, b)$$
Donc : 
$$m(u'k + vq') = ppcm(a, b)$$
Donc $m | ppcm(a, b)$. \\

\section{Propriétés du $ppcm$}
\begin{tcolorbox}[title=Propostion 12.46, title filled=false, colframe=lightblue, colback=lightblue!10!white]
    Soit $a$ et $b$ deux entier non nuls, alors : 
    \begin{enumerate}
        \item pour tout $n \in \mathbb{Z}$, si $a|n$ et $b|n$, alors $ppcm(a,b)|n$;
        \item si $a$ et $b$ sont premiers entre eux, alors $ppcm(a,b) = |ab|$;
        \item pour tout $k \in \mathbb{N}^*$, $ppcm(ka, kb) = kppcm(a,b)$;
        \item $ppcm(a,b) \times pgcd(a,b) = |ab|$;
        \item pour tout $n \in \mathbb{N}$, $ppcm(a^n, b^n) = ppcm(a,b)^n$.
    \end{enumerate}
\end{tcolorbox}

\begin{enumerate}
    \item RAF $\text{(12.44)}$
    \item On suppose que $a >0$ et $b > 0$. \\
    $$ab =ba$$
    avec $a \wedge b = 1$. \\
    D'après $\text{(12.45)}$ :
    $$ppcm(a,b) = ab$$

    \item On écrit $\text{(12.45)}$ :
    $$ppcm(a,b) = au = bv \text{ (avec $u \wedge v = 1$)}$$
    Alors : 
    \begin{align*}
        b \wedge ppcm(a,b) &= (ak)u \\
        &= (bk)v
    \end{align*}
    Donc $\text{(12.45)}$ :
    $$ppcm(ak, bk) = kppcm(a,b)$$

    \setcounter{enumi}{4}
    \item Avec les mêmes notations : 
    \begin{align*}
        ppcm(a, b)^n &= a^n u^n \\
        &= b^n v^n \text{ (avec $u^n \wedge v^n = 1$)}
    \end{align*}
    Donc $\text{(12.45)}$ :
    $$ppcm(a^n, b^n) = ppcm(a,b)^n$$
    
    \setcounter{enumi}{3}
    \item D'après $\text{(12.39)}$ (avec $a > 0$ et $b > 0$) :
    \begin{align*}
        a &= pgcd(a,b)u \\
        b &= pgcd(a,b)v \text{ (avec $u \wedge v = 1$)} \\
        pgcd(a, b) \times ppcm(a, b) &= pgcd(a, b) ppcm(pgcd(a, b)u, pgcd(a, b)v) \\
        &\underset{\text{(3.)}}{=} pgcd(a, b)^2 ppcm(u, v) \\
        &\underset{\text{(2.)}}{=} pgcd(a, b)^2uv \\
        &= ab
    \end{align*}
\end{enumerate}

\setcounter{section}{49}
\section{Propriétés}
\begin{tcolorbox}[title=Propostion 12.50, title filled=false, colframe=lightblue, colback=lightblue!10!white]
    \begin{enumerate}
        \item Si $p \in \mathbb{P}$, alors pour tout $n \in \mathbb{Z}$, soit $p|n$ soit $pgcd(n, p) = 1$. 
        \item Si $n \geq 2$, alors $n$ possède au moins un diviseur premier. 
        \item L'ensemble $\mathbb{P}$ est infini. 
        \item Si $n > 1$ n'as pas de diviseur dans $\left[ 2 ; \sqrt{n} \right]$, alors $n$ est premier. 
        \item Si $p \in \mathbb{P}$, alors pour tout $a$ et $b$ entiers, on a $(a + b)^p \equiv a^p + b^p \pmod p$.	
    \end{enumerate}
\end{tcolorbox}

\begin{enumerate}
    \item On suppose que $p \not | n$. \\
    Soit $d \in \mathcal{D}_p \cap \mathcal{D}_{n}$. \\
    $d > 0$ et $d \neq p$. \\
    Donc $d = 1$. \\
    Donc $p \wedge n = 1$.

    \item On raisonne par récurrence forte $\rightarrow$ cf. $\text{(2.41)}$. 
    
    \item On suppose par l'absurde que : 
    $$\mathbb{P} = \{p_1, p_2, \ldots, p_n\}$$
    On pose : 
    $$m = \prod_{i = 1}^{n} (p_i) + 1$$
    Soit $p_i \in \mathbb{P}$ tel que $p_i | m \text{ (12.50.2)}$. \\
    Donc $p_i | 1$. \\
    Absurde. 

    \item On suppose $n \not \in \mathbb{P}$. \\
    Soit $n = ab$ avec $a \geq 2$ et $b \geq 2$. \\
    Si $a > \sqrt{n}$ et $b > \sqrt{n}$, alors $ab = n > \sqrt{n}^2 = n$. \\
    Absurde.

    \item D'après le binôme de Newton : 
    \begin{align*}
        (a + b)^p &= \sum_{k = 0}^{p} \binom{p}{k} a^{k}b^{p-k} \\
        &= a^p + b^p + \sum_{k = 1}^{p-1} \binom{p}{k} a^{k}b^{p-k} 
    \end{align*}
    Or, pour $k \in \llbracket 1 ; p-1 \rrbracket, p \binom{p - 1}{k - 1} = k \binom{p}{k} \text{ (formule du capitaine)}$.  \\
    Or $k \wedge p = 1$ et $p \left| p \binom{p-1}{k-1} \right.$ soit $p \left| \binom{p}{k} \right.$. \\
    Donc : 
    $$ p \left| \binom{p}{k} \right.$$ 
    Donc :
    $$(a + b)^p \equiv a^p + b^p \pmod p$$
\end{enumerate}

\section{Petit théorème de Fermat}
\begin{tcolorbox}[title=Théorème 12.51, title filled=false, colframe=orange, colback=orange!10!white]
    Pour tout $n \in \mathbb{Z}$ et $p \in \mathbb{P}$, on a $n^p \equiv n \pmod p$. En outre, si $pgcd(n,p) = 1$, alors $n^{p-1} \equiv 1 \pmod p$.
\end{tcolorbox}

Soit $p \in \mathbb{P}$. On montre le résultat pour $n \geq 0$ par récurrence. \\
On a bien $0^p = 0 \equiv 0 \pmod{p}$. 
Si $n^p \equiv n \pmod{p}$, alors  :
\begin{align*}
    (n + 1)^p &\equiv n^p + 1^p \pmod{p} \text{ (12.50.5)}. \\
    &\equiv n + 1 \pmod{p} \text{ (Hypothèse de récurrnce)}
\end{align*}
Soit $n \in \mathbb{N}$. \\
\begin{itemize}
    \item Si $p \geq 3$ (donc $p$ est impair), alors : 
    \begin{align*}
        n^p &\equiv n \pmod{p} \\
        (-n)^p &\underset{p \text{ impair}}{\equiv} -n^p \pmod{p} \\
        &\equiv -n \pmod{p}
    \end{align*}

    \item Si $p = 2$, $-1 \equiv 1 \pmod{2}$. \\
    Donc : 
    \begin{align*}
        (-n)^2 &\equiv n^2 \pmod{2} \\
        &\equiv n \pmod{2} \\
        &\equiv -n \pmod{2}
    \end{align*}
\end{itemize}

\section{Décomposition en produit de facteurs premiers}
\begin{tcolorbox}[title=Théorème 12.52, title filled=false, colframe=orange, colback=orange!10!white]
    Soit $n \in \mathbb{Z} \backslash \{ -1, 0, 1 \}$, alors il existe des nombres premiers $p_1, \ldots, p_r$ tous distincts, et $(\alpha_1, \ldots, \alpha_r) \in \left( \mathbb{N}^* \right)^r$ et $\epsilon \in \{ \pm 1 \}$ tels que 
    $$n = \epsilon p_1^{\alpha_1} \times \dots \times p_r^{\alpha_r}$$
    Cette décomposition est unique à l'ordre près. 
\end{tcolorbox}

\noindent \underline{Existence :} \\
On montre l'existence par récurrence forte sur $\mathbb{N} \backslash \{ 0, 1 \}$. 
\begin{itemize}
    \item RAF si $n = 2$. 
    \item On suppose le résultat vrai pour tout $k \in \llbracket 2 ; n \rrbracket$. \\
    \begin{itemize}
        \item Si $n + 1 \in \mathbb{P}$ : RAF
        \item Si $n + 1 \not \in \mathbb{P}$, on écrit : 
        $$n + 1 = k \times q \text{ avec $(k, q) \in \llbracket 2, n \rrbracket^2$}$$
        Donc $k$ et $q$ sont des produits de facteurs premiers. \\
        Donc $n+1 = kq$ est aussi un produit de facteurs premiers. \\
        Le résultat est donc vrai pour tout $n \in \mathbb{N}$ et par extension pour $-n \text{ ($\epsilon = -1$)}$. 
    \end{itemize}
\end{itemize}

\noindent \underline{Unicité :} \\
On suppose que : 
$$n = \epsilon p_1^{\alpha_1} \times \dots \times p_r^{\alpha_r} = \epsilon' q_1^{\beta_1} \times \dots \times q_s^{\beta_s}$$
Nécessairement, $\epsilon = \epsilon'$. \\
Soit $p_i \in \{ p_1, \ldots, m_r \}$. \\
On a $p_i | n$ donc $p_i \left| q_1^{\beta_1} \times \dots \times q_s^{\beta_s} \right.$. \\
Il existe $p_i \in \mathbb{P}$ donc $j \in \llbracket 1 ; s \rrbracket$ tel que $p_i | q_j$. \\
Donc $p_i = \underbrace{q_j}_{\in \mathbb{P}}$. \\
Ainsi : 
$$\left\{ p_1, \ldots, p_r \right\} \subset \left\{ q_1, \ldots, q_s \right\}$$
Par symétrie : 
$$\left\{ p_1, \ldots, p_r \right\} = \left\{ q_1, \ldots, q_s \right\}$$
Donc $r = s$ et quitte à renommer $q_j$, on peut supposer que : \\
$$\forall i \in \llbracket 1 ; r \rrbracket, p_i = q_i$$
\begin{align*}
    p_i^{\alpha_i} \left | n \right. &\text{ donc } p_i^{\alpha_i} \left | \prod_{j = 1}^{r} p_j^{\beta_j} \right. \\
&\text{ donc } \alpha_i \leq \beta_i
\end{align*}
Par symétrie, $\alpha_i = \beta_i$. \\
L'unicité est prouvée. 

\setcounter{section}{53}
\section{Caractérisation de la valuation}
\begin{tcolorbox}[title=Théorème 12.54, title filled=false, colframe=orange, colback=orange!10!white]
    Soit $n \in \mathbb{Z}^*$ et $p \in \mathbb{P}$ et $d \in \mathbb{N}$. Alors $d = v_p(n)$ si et seulement si $n = p^d u$, avec $u \wedge p = 1$.
\end{tcolorbox}

On a : 
\begin{align*}
    d = v_p(n) &\Leftrightarrow (p^d | n \text{ et } p^{d+1} \not | n) \\
    &\Leftrightarrow \exists u \in \mathbb{Z}, n = p^d u \text{ et } p^{d+1} \not | u \\
    &\Leftrightarrow \exists u \in \mathbb{Z}, n = p^d u \text{ et } p \not | u \\
    &\underset{\text{($p \in \mathbb{P}$)}}{\Leftrightarrow} \exists u \in \mathbb{Z}, n = p^d u \text{ et } u \wedge p = 1
\end{align*}

\section{Valuation et décomposition en produit de facteurs premiers}

\begin{tcolorbox}[title=Théorème 12.55, title filled=false, colframe=orange, colback=orange!10!white]
    Si $p | n$, alors $v_p(n)$ est la puissance de $p$ intervenant dans la décomposition en produit de facteurs premiers de $n$. 
\end{tcolorbox}

On écrit la décomposition : 
\begin{align*}
    n = \epsilon \prod_{i=1}^{r} p_i^{\alpha_i}
\end{align*}
Soit $k \in \llbracket 1, r \rrbracket$. 
\begin{align*}
    n = \epsilon \times p_k^{\alpha_k} \times \underbrace{\prod_{i \neq k} p_i^{\alpha_i}}_{:= u \text{ (avec $u \wedge p_k = 1$)}}
\end{align*}
Donc $\text{(12.54)}$ : 
$$\boxed{v_{p_k}(n) = \alpha_k}$$

\section{Propriétés de la valuation}

\begin{tcolorbox}[title=Propostion 12.56, title filled=false, colframe=lightblue, colback=lightblue!10!white]
    Pout tout $(n, m) \in \mathbb{Z}^2$ et $p \in \mathbb{P}$, on a
    \begin{enumerate}
        \item $p|n$ si et seulement si $v_p(n) > 0$;
        \item $v_p(mn) = v_p(m) + v_p(n)$;
        \item $v_p(n + m) \geq \min(v_p(n), v_p(m))$ avec égalité si les valuations sont distinctes;
        \item $n|m \Leftrightarrow (\forall q \in \mathbb{P}, v_q(n) \leq v_q(m))$;
        \item si de plus $n$ et $m$ sont non nuls alors
        $$v_p(n \wedge m) = \min(v_p(n), v_p(m)) \text{ et } v_p(n \vee m) = \max(v_p(n), v_p(m)).$$
    \end{enumerate}
\end{tcolorbox}

\begin{enumerate}
    \item RAF
    
    \item On écrit $m = p^{v_p(m)} \times u$ et $n = p^{v_p(n)} \times v$ avec $u \wedge p = 1 = v \wedge p$ $\text{(12.54)}$. \\
    Donc $mn = p^{v_p(m) + v_p(n)} \times uv$. \\
    Or $p \wedge (uv) = 1$. \\
    Donc $\text{(12.54)}$ :
    $$\boxed{v_p(mn) = v_p(m) + v_p(n)}$$

    \item On suppose que $v_p(m) \leq v_p(n)$. \\
    Ainsi : 
    \begin{align*}
        n + m &= p^{v_p(n)} \times v + p^{v_p(m)} \times u \\
        &= p^{v_p(m)} \left[ u + v_p^{v_p(n) - v_p(m)} \right] \\
    \end{align*}
    Ainsi, $p^{v_p(m)} | n + m$. \\
    Par définition :
    $$\boxed{v_p(m + n) \geq v_p(m) = \min(v_p(m), v_p(n))}$$. \\
    Si on suppose de plus que $v_p(m) \neq v_p(n)$, alors 
    $$p \wedge (u + v \times p^{v_p(n) - v_p(m)}) = p \wedge u = 1$$
    Donc $\text{(12.54)}$ :
    $$\boxed{v_p(n + m) = v_p(m) = \min(v_p(m), v_p(n))}$$

    \item On a :
    \begin{align*}
        n|m &\text{ ssi } \text{la décomposition en produit de facteurs premiers de $n$ se retrouve dans celle de $m$. } \\
        &\underset{\text{(12.55)}}{\text{ ssi }} \text{pour tout $p \in \mathbb{P}$ tel que $p|n$, alors $v_p(n) \leq v_p(m)$. } \\
        &\underset{\text{(si $p \not | n, v_p(n) = 0 \leq v_p(m)$)}}{\text{ ssi }} \text{pour tout $\boxed{p \in \mathbb{P}, v_p(n) \leq v_p(m)}$. }
    \end{align*}

    \item On a $(n \wedge m) | n$ et $(n \wedge m) | m$. \\
    Donc $\text{(12.56.4)}$ $\boxed{v_p(n \wedge m) \leq \min(v_p(n), v_p(m))}$. \\
    On suppose par exemple que $v_p(n) \leq v_p(m)$. \\
    Donc $p^{v_p(n)} | n$ et $p^{v_p(n)} | m$. \\
    Donc $p^{v_p(n)} | n \wedge m$. \\
    Par définition $\boxed{v_p(n \wedge m) \geq v_p(n)}$. \\
    Donc : 
    $$\boxed{v_p(n \wedge m) = \min(v_p(n), v_p(m))}$$
    \\
    On rappelle que $(n \wedge m) \times (n \vee m) = |nm|$. \\
    Donc $v_p((n \wedge m) \times (n \vee m)) = v_p(nm)$. \\
    Donc $\text{(12.56.2)}$ :
    \begin{align*}
        v_p(n \vee m) &= v_p(n) + v_p(m) - v_p(n \wedge m) \\
        &= v_p(n) + v_p(m) - \min(v_p(n), v_p(m)) \\
        &= \boxed{\max(v_p(n), v_p(m))}
    \end{align*}
\end{enumerate}
Les preuves ont été rédigées avec les hypothèses $n \neq 0$ et $m \neq 0$. Si l'un des entiers est nul, on vérifie les assertions avec la convention $v_p(0) = +\infty$.
\end{document}