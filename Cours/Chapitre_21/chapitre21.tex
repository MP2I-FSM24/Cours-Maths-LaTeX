\documentclass[../main.tex]{subfiles}

\begin{document}
\setcounter{chapter}{20}
\chapter{Applications linéaires}
\tableofcontents
\clearpage

\setsection{3}
\section{Exemple}
\begin{tcolorbox}[title=Exemple 21.4.1, title filled=false, colframe=darkgreen, colback=darkgreen!10!white]
    L'application de $\mathbb{R}^2$ dans $\mathbb{R}$ définie par $f(x, y) = 2x + 3y$. 
\end{tcolorbox}

\noindent Soit $((x, y), (x', y'), \lambda) \in (\mathbb{R}^2)^2 \times \mathbb{R}$. On a
\begin{align*}
    f((x, y) + \lambda (x', y')) &= f(x + \lambda x', y + \lambda y') \\
    &= 2(x + \lambda x') + 3(y + \lambda y') \\
    &= 2x + 3y + \lambda(2x' + 3y') \\
    &= f(x, y) + \lambda f(x', y').
\end{align*}

\setsection{7}
\section{Structure de $\mathcal{L}(E, F)$}
\begin{tcolorbox}[title=Propostion 21.8, title filled=false, colframe=lightblue, colback=lightblue!10!white]
    $\mathcal{L}(E, F)$ est un estpace vectoriel sur $\mathbb{K}$.
\end{tcolorbox}

\noindent \begin{itemize}
    \item $\mathcal{L}(E, F) \subset F^E$
    \item $\overline{0} \mathcal{L}(E, F)$
    \item Soit $(f, g) \in \mathcal{L}(E, F)^2$ et $\alpha \in \mathbb{K}$. Soit $(x, y) \in E^2, \lambda \in \mathbb{K}$. On a : 
    \begin{align*}
        (f + \alpha g)(x + \lambda y) &= f(x + \lambda y) + \alpha g(x + \lambda y) \\
        &= f(x) + \lambda f(y) + \alpha g(x) + \alpha \lambda g(y) \\
        &= f(x) + \alpha g(x) + \lambda (f(y) + \alpha g(y)) \\
        &= (f + \alpha g)(x) + \lambda (f + \alpha g)(y).
    \end{align*}
    Donc $f + \alpha g \in \mathcal{L}(E, F)$.
\end{itemize}

\setsection{9}
\section{Composition de deux AL}
\begin{tcolorbox}[title=Propostion 21.10, title filled=false, colframe=lightblue, colback=lightblue!10!white]
    Soit $f \in \mathcal{L}(E, F)$ et $g \in \mathcal{L}(F, G)$, alors $g \circ f \in \mathcal{L}(E, G)$.
\end{tcolorbox}

\noindent Soit (x, y) $\in E^2$ et $\lambda \in \mathbb{K}$ : 
\begin{align*}
    g \circ f(x + \lambda y) &= g(f(x + \lambda y)) \\
    &= g(f(x) + \lambda f(y)) \\
    &= g(f(x)) + \lambda g(f(y)) \\
    &= g \circ f(x) + \lambda g \circ f(y).
\end{align*}
Donc $g \circ f \in \mathcal{L}(E, G)$.


\end{document}