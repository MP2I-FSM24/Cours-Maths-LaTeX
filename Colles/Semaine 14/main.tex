\documentclass[titlepage, twoside]{report}

\usepackage[utf8]{inputenc}
\usepackage[T1]{fontenc}
\usepackage{geometry}
\usepackage{subfiles}
\usepackage[french]{babel}
\usepackage{amsmath}
\usepackage{amssymb}
\usepackage{stmaryrd}
\usepackage{dsfont}
\usepackage{tikz}
\usepackage{cancel}
\usepackage{fancyhdr}
\usepackage{tcolorbox}
\usepackage[dvipsnames]{xcolor}

\geometry{a4paper, left=20mm, right=20mm, top=20mm, bottom=20mm}

\setcounter{tocdepth}{3}

\definecolor{darkgreen}{RGB}{6, 64, 43}
\definecolor{lightblue}{RGB}{87, 185, 255}

\pagestyle{fancy}
\fancyhead
\fancyfoot

\fancyhead[L]{Programme de colle}
\fancyhead[R]{Axel Montlahuc}
\fancyfoot[C]{\thepage}

\makeatletter
\renewcommand{\tableofcontents}{%
  \@starttoc{toc}
}
\makeatother
\renewcommand{\thesection}{\Roman{section}}

\begin{document}
\chapter*{Programme de colle : semaine 13}
\tableofcontents


\section{Limites et continuité}
\subsection{Questions de cours}
\subsubsection{Enoncer et démontrer le théorème de caractérisation séquentielle de la limite d'une fonction}
\begin{tcolorbox}[title=Théorème 15.34, title filled=false, colframe=orange, colback=orange!10!white]
    Soit $f:X\to \mathbb{R}$ une fonction et $a \in \overline{X}$ et $\ell \in \overline{\mathbb{R}}$. Sont équivalentes : 
    \begin{enumerate}
        \item $\lim\limits_a f = \ell \Leftrightarrow \forall u_n \to a, \lim f(u_n) = \ell \text{ ($= f(\lim u_n)$)}$
        \item Pour toute suite $(u_n)$ de limite $a$ à valeurs dans $X$, la suite $(f(u_n))$ a pour limite $\ell$. 
    \end{enumerate}
\end{tcolorbox}

$\boxed{1 \Rightarrow 2}$ \\
On suppose que $\lim\limits_a f = \ell$. \\
Soit $(u_n) \in X^{\mathbb{N}}$ avec $u_n \underset{n \to +\infty}{\longrightarrow} a$. \\
Soit $V \in \mathcal{V}(\ell)$. On choisit $U \in \mathcal{V}(a)$ tel que : 
\begin{align*}
    f(U \cap X) \subset V \text{ ($\lim\limits_a f = \ell$)}
\end{align*}
Comme $u_n \underset{n \to +\infty}{\longrightarrow} a$, on choisit $N \in \mathbb{N}$ tel que : 
\begin{align*}
    \forall n \geq N, u_n \in U \cap X
\end{align*}
Donc : 
\begin{align*}
    \forall n \geq N, f(u_n) \in V
\end{align*}
Donc : 
\begin{align*}
    f(u_n) \underset{n \to +\infty}{\longrightarrow} \ell
\end{align*} \\

$\boxed{1 \Leftarrow 2}$ \\
Par contraposée. On suppose que $f$ n'admet pas $\ell$ comme limite en $a$. Pour tout $n \in \mathbb{N}$, on note : 
\begin{align*}
    V_n = \begin{cases}
        ]a - \frac{1}{n+1}, a + \frac{1}{n+1}[ \text{ si $a \in \mathbb{R}$} \\
        [n, +\infty[ \text{ si $a = +\infty$} \\
        ]-\infty, -n] \text{ si $a = -\infty$}
    \end{cases}
\end{align*}
Par définition, il existe $W \in \mathcal{V}(\ell)$ tel que pour tout $V \in \mathcal{V}(a)$, il existe $x \in V \cap X$ et $f(x) \neq W$. \\
Pour tout $n \in \mathbb{N}$, on choisit $x_n \in V_n \cap X$ tel que $f(x_n) \neq W$. \\
Par construction : 
\begin{align*}
    (x_n) \in X^\mathbb{N}, x_n \underset{n \to +\infty}{\longrightarrow} a \text{ et } f(x_n) \cancel{\underset{n \to +\infty}{\longrightarrow}} \ell
\end{align*}

\subsubsection{Enoncer et démontrer le théorème de Heine}
\begin{tcolorbox}[title=Théorème 15.65, title filled=false, colframe=orange, colback=orange!10!white]
    Une fonction continue sur un segment est uniformément continue sur ce segment. 
\end{tcolorbox}

\noindent \underline{Rappel :}
\begin{align*}
    C^0(I) &: \forall x \in I, \forall \epsilon > 0, \exists \eta > 0, \forall y \in I, |x-y| < \eta \Rightarrow |f(x) - f(y)| < \epsilon \\
    Cu(I) &: \forall \epsilon > 0, \exists \eta > 0, \forall (x,y) \in I^2, |x-y| < \eta \Rightarrow |f(x) - f(y)| < \epsilon
\end{align*}
On raisonne par l'absurde. Soit $f$ continue sur $[a,b]$ mais non uniformément continue sur $[a,b]$. \\
On choisit $\epsilon$ tel que : 
\begin{align*}
    \forall \eta > 0, \exists (x, y) \in [a,b]^2, |x - y| < \eta \text{ et } |f(x) - f(y)| \geq \epsilon
\end{align*}
Ainsi, pour tout $b \in \mathbb{N}^*$, on choisit un couple $(x_n, y_n) \in [a,b]^2$ tel que : 
\begin{align*}
    |x_n - y_n| < \frac{1}{n} \text{ et } \underbrace{|f(x_n) - f(y_n)|}_{(*)} \geq \epsilon
\end{align*}
En particulier $(x_n)$ est bornée donc d'après le théorème de Bolzano-Weierstrass, on en extrait $(x_{\varphi(n)})$ suite convergente vers $\ell$. \\
D'après le TCILPPL, $\ell \in [a,b]$. \\
Comme : 
\begin{align*}
    \forall n \in \mathbb{N}, |x_{\varphi(n)} - y_{\varphi(n)}| < \frac{1}{\varphi(n)} \underset{n \to +\infty}{\longrightarrow} 0
\end{align*}
Alors : 
\begin{align*}
    y_{\varphi(n)} \underset{n \to +\infty}{\longrightarrow} \ell
\end{align*}
Par continuité : 
\begin{align*}
    f(x_{\varphi(n)}) \underset{n \to +\infty}{\longrightarrow} f(\ell) \text{ et } f(y_{\varphi(n)}) \underset{n \to +\infty}{\longrightarrow} f(\ell)
\end{align*}
Donc par opération : 
\begin{align*}
    |f(x_{\varphi(n)}) - f(y_{\varphi(n)})| \underset{n \to +\infty}{\longrightarrow} 0
\end{align*}
Absurde d'après $(*)$. 

\subsubsection{Démontrer que l'image continue d'un compact est compact. Démontrer qu'une fonction continue sur un intervalle est injective si et seulement si elle est strictement monotone}
\begin{tcolorbox}[title=Lemme 15.68, title filled=false, colframe=orange, colback=orange!10!white]
    L'image continue d'un compact est compact. 
\end{tcolorbox}

\noindent Soit $I$ un segment, donc un intervalle. \\
Comme $f$ est continue sur $I$, $f(I)$ est un intervalle (TVI v3). \\
Montrons que $f(I)$ est compact. \\
Soit $(y_n) \in f(I)^{\mathbb{N}}$. Pour tout $n \in \mathbb{N}$, soit $x_n \in I$ tel que : 
\begin{align*}
    y_n = f(x_n)
\end{align*}
Or $I$ est compact (15.67), on choisit : 
\begin{align*}
    x_{\varphi(n)} \underset{n \to +\infty}{\longrightarrow} \ell \in I
\end{align*}
$y_{\varphi(n)} \underset{n \to +\infty}{\longrightarrow} f(\ell)$ car $f$ est continue sur $I$. 

\begin{tcolorbox}[title=Théorème 15.72, title filled=false, colframe=orange, colback=orange!10!white]
    Soit $I$ un intervalle et $f$ une fonction continue sur $I$. Alors $f$ est injective si et seulement si $f$ est strictement monotone. 
\end{tcolorbox}

$\boxed{\Leftarrow}$ \\
RAS \\ \\

$\boxed{\Rightarrow}$ \\
Supposons $f$ non strictement monotone. \\
On peut supposer qu'il existe alors : 
\begin{align*}
    x < y < z
\end{align*}
tels que $f(x) < f(y)$ et $f(z) < f(y)$. \\
Soit :
\begin{align*}
    \lambda = \frac{f(y) + \max(f(y), f(z))}{2} &\in ]f(x), f(y)[ \\
    &\in ]f(z), f(y)[
\end{align*}
Par continuité de $f$ sur les intervalles $]x, y[$ et $]y, z[$, il existe $\alpha \in ]x, y[$ et $\beta \in ]y, z[$ tels que : 
\begin{align*}
    f(\alpha) = \lambda = f(\beta)
\end{align*}
Donc $f$ n'est pas injective. 


\subsection{Exercices types}
\begin{tcolorbox}[title=Exercice 1, title filled=false, colframe=darkgreen, colback=darkgreen!10!white]
    Soit $f$ et $g$ dans $\mathcal{C}(\mathbb{R}, \mathbb{R})$ telles que $\left. f \right|_{\mathbb{Q}} = \left. g \right|_{\mathbb{Q}}$. Montrer alors que $f$ et $g$ sont égales sur $\mathbb{R}$ tout entier. 
\end{tcolorbox}

\begin{tcolorbox}[title=Exercice 2, title filled=false, colframe=darkgreen, colback=darkgreen!10!white]
    Soit $f \in \mathcal{C}([a, b], \mathbb{R})$. On suppose que
    $$f([a, b]) \subset [a, b] \text{ ou } [a, b] \subset f([a, b])$$
    Montrer qu'alors $f$ possède un point fixe. 
\end{tcolorbox}

\begin{tcolorbox}[title=Exercice 3, title filled=false, colframe=darkgreen, colback=darkgreen!10!white]
    Soit $(f, g) \in \mathcal{C}([a, b], \mathbb{R})$. On suppose que
    $$f([a, b]) \subset [a, b] \text{ et } g([a, b]) \subset [a, b] \text{ et } f \circ g = g \circ f$$
    \begin{enumerate}
        \item Montrer que si $f > g$ sur l'intervalle $[a, b]$, il existe $K > 0$ tel que pour tout $n \in \mathbb{N}, f^n \geq g^n + nK$ sur $[a, b]$, où $f^n$ désigne la composée. 
        \item En déduire que les graphes de $f$ et $g$ possèdent un point d'intersection. 
    \end{enumerate}
\end{tcolorbox}

\section{Arithmétique des polynômes}
\subsection{Questions de cours}
\subsubsection{Enoncer et démontrer le théorème de la division euclidienne sur $\mathbb{K}[X]$}
\begin{tcolorbox}[title=Théorème 16.1, title filled=false, colframe=orange, colback=orange!10!white]
    Soit $A \in \mathbb{K}[X]$ et $B \in \mathbb{K}[X]$ non nul, il existe un unique couple de polynômes $(Q,R)$ tel que $A = BQ + R$ avec $\deg R < \deg B$. Le polynôme $Q$ est appelé \textbf{quotient} et $R$ le \textbf{reste}. 
\end{tcolorbox}

\noindent \underline{Existence :} \\
On raisonne par récurrence sur le degré de $A$. \\
\begin{itemize}
    \item Pour $n = \deg A = 0$. Soit $A \in \mathbb{K}[X]$.
    \begin{itemize}
        \item Si $\deg B > 0$, alors $(0, A)$ convient. \\
        \item Si $\deg B = 0$, le couple $(B^{-1} \times A, 0)$ convient (comme $B$ est constant et non nul), alors $B \in \mathbb{K}^*$ donc inversible). \\
    \end{itemize}

    \item On suppose le résultat vrai pour tout $A \in \mathbb{K}_n[X]$. \\
    Soit $A \in \mathbb{K}_{n+1}[X]$ avec $\deg A = n+1$. \\
    On écrit $A = \underbrace{a}_{\neq 0} X^{n+1} + A_1$ avec $A_1 \in \mathbb{K}_n[X]$. 
    \begin{itemize}
        \item Si $\deg A < \deg B$, le couple $(0, A)$ convient. 
        \item Si $\deg A \geq \deg B$ et on note $b$ le coefficient dominant de $B$ : 
        \begin{align*}
            A - ab^{-1} B \times X^{n+1 - \deg B} \in \mathbb{K}_n[X]
        \end{align*}
        D'après l'hypothèse de récurrence, on choisit $(Q, R) \in \mathbb{K}[X]^2$ tel que $\deg R < \deg B$ et $A - ab^{-1} B \times X^{n+1 - \deg B} = QB + R$. \\
        Donc : 
        \begin{align*}
            A = \left[ Q + ab^{-1}X^{n+1 - \deg A} \right] \times B + R
        \end{align*}
    \end{itemize}
\end{itemize}

\noindent\underline{Unicité :} \\
On suppose que $A = BQ + R = BQ_1 + R_1$. \\
Donc : 
\begin{align*}
    B(Q - Q_1) &= R_1 - R \\
    \text{donc } \underbrace{\deg{(B(Q - Q_1))}}_{\deg{B} + \deg{Q - Q_1}} &= \deg{(R_1 - R)} \\
    &\leq \max(\deg{R_1}, \deg{R}) \\
    &< \deg{B} \\
    \text{donc } \deg{(Q - Q_1)} &< 0 \\
    \text{donc } Q - Q_1 &= 0 \\
    \text{puis } R_1 - R = 0
\end{align*}

\subsubsection{Enoncer et démontrer le théorème de principalité dans $\mathbb{K}[X]$}
\begin{tcolorbox}[title=Théorème 16.15, title filled=false, colframe=orange, colback=orange!10!white]
    Soit $I$ un idéal de $\mathbb{K}[X]$ non réduit à $\{0\}$. Il existe un unique polynôme unitaire $D$ tel que
    $$I = D \mathbb{K}[X]$$
\end{tcolorbox}

\noindent \underline{Existence :} \\
Soit $I \neq \{0\}$ un idéal. \\
On note $A = \{ \deg P, P \in I\backslash \{0\} \} \subset \mathbb{N}$. \\
$A \neq \emptyset$ ($I \neq \{0\}$), d'après la propriété fondamentale de $\mathbb{N}$, $A$ possède un plus petit élément noté $n \geq 0$. \\
Comme $n \in A$, on choisit $D \in I$ tel que $\deg D = n$. \\
Comme $I$ est un idéal de $\mathbb{K}[X]$ et que $\mathbb{K} = \mathbb{K}_0[X] \subset \mathbb{K}[X]$, on a : 
\begin{align*}
    \forall \alpha \in \mathbb{K}, \alpha D \in I
\end{align*}
On peut donc supposer $D$ unitaire. 
Comme $I$ est un idéal de $\mathbb{K}[X]$, on a : 
\begin{align*}
    D \times \mathbb{K}[X] \subset I
\end{align*}
Soit $P \in I$. On effectue la division euclidienne de $P$ par $D$ ($\neq 0$) : 
\begin{align*}
    P = BD + R
\end{align*}
avec $\deg R \subset \deg D$. \\
Or : 
\begin{align*}
    R &= \underbrace{P}_{\in I} - \underbrace{BD}_{\in I} \\
    &\in I
\end{align*}
Par définition de $\deg D = n$, $R = 0$. \\ \\

\noindent \underline{Unicité :} \\
\begin{align*}
    I = D \mathbb{K}[X] = J \mathbb{K}[X] \\
\end{align*}
avec $D$ et $J$ unitaires. \\
Or ils sont associés, donc égaux. 

\subsubsection{Enoncer et démontrer la caractérisation des PGCD par les idéaux de $\mathbb{K}[X]$}
\begin{tcolorbox}[title=Propostion 16.18, title filled=false, colframe=lightblue, colback=lightblue!10!white]
    Soit $A$ et $B$ deux polynômes non tous deux nuls. Soit $D \in \mathbb{K}[X]$. Alors $D$ est un PGCD de $A$ et $B$ si et seulement si 
    $$A \mathbb{K}[X] + B \mathbb{K}[X] = D \mathbb{K}[X].$$
\end{tcolorbox}

\noindent D'après (16.15), on choisit $F \in \mathbb{K}[X]$ tel que :
\begin{align*}
    A \mathbb{K}[X] + B \mathbb{K}[X] = F \mathbb{K}[X]
\end{align*}
Soit $D \in \mathbb{K}[X]$. \\ \\

$\boxed{\Rightarrow}$ \\
On suppose que $D$ est un PGCD. \\
Donc $D|A$ et $D|B$. \\
Donc $D|F \text{ (combinaison $F \in A \mathbb{K}[X] + B \mathbb{K}[X]$)}$. \\
Or $F|A$ et $F|B$ ($A \in F \mathbb{K}[X]$, $B \in F \mathbb{K}[X]$). \\
Par maximalité de $\deg D$, on a $F$ et $D$ associés. \\ \\

$\boxed{\Leftarrow}$ \\
\begin{align*}
    D \mathbb{K}[X] = A \mathbb{K}[X] + B \mathbb{K}[X] = F \mathbb{K}[X]
\end{align*}
Donc $D|A$ et $D|B$. \\
Pour tout diviseur commun $P$ de $A$ et $B$, $P|A$ et $P|B$. \\
Donc $P|D$ ($D \in A \mathbb{K}[X] + B \mathbb{K}[X])$. \\
Donc $\deg D$ est maximal pour la divisibilité. 

\subsection{Exercices types}
\begin{tcolorbox}[title=Exercice 1, title filled=false, colframe=darkgreen, colback=darkgreen!10!white]
    Soit $\left(P_n\right)_{n \in \mathbb{N}}$ la suite de polynômes définie par les relations
    $$P_0=0, P(1)=1 \text { et } \forall n \in \mathbb{N}, P_{n+2}=X P_{n+1}-P_n$$
    \begin{enumerate}
        \item Déterminer $P_2$ et $P_3$.
        \item Pour tout $n \in \mathbb{N}^*$, déterminer le degré et le coefficient dominant de $P_n$.
        \item Montrer que pour tout $n \in \mathbb{N}, P_{n+1}^2=1+P_n P_{n+2}$.
        \item En déduire que pour tout $n \in \mathbb{N}, P_n$ et $P_{n+1}$ sont premiers entre eux.
        \item Montrer que pour tout $m \in \mathbb{N}$ et pour tout $n \in \mathbb{N}^*$, on a
        $$P_{m+n}=P_n P_{m+1}-P_{n-1} P_m$$
        \item Montrer que pour tout $m \in \mathbb{N}$ et tout $n \in \mathbb{N}^*$, on a
        $$P_{m+n} \wedge P_n=P_n \wedge P_m$$
        En déduire que
        $$P_m \wedge P_n=P_n \wedge P_r$$
        où $r$ est le reste de la division euclidienne de $m$ par $n$.
        \item Conclure que pour tout $m \in \mathbb{N}$ et tout $n \in \mathbb{N}^*$, on a
        $$P_n \wedge P_m=P_{n \wedge m}.$$
    \end{enumerate}
\end{tcolorbox}

\begin{tcolorbox}[title=Exercice 2, title filled=false, colframe=darkgreen, colback=darkgreen!10!white]
    Calculer le reste de la division euclidienne de $X^n$ par $(X-1)^4$ pour tout $n \geq 4$.
\end{tcolorbox}

\begin{tcolorbox}[title=Exemple 3, title filled=false, colframe=darkgreen, colback=darkgreen!10!white]
    Soit $\theta \in \mathbb{R}$ et $n \in \mathbb{N}^*$. On note $P$ le polynôme $(X+1)^n-\mathrm{e}^{2 \mathrm{in} n}$.
    \begin{enumerate}
        \item Déterminer les racines de $P$ dans $\mathbb{C}$.
        \item En déduire que $P$ est scindé à racines simples sur $\mathbb{C}$.
        \item Simplifier le produit $\prod\limits_{k=0}^{n-1} \sin \left(\theta+\frac{k \pi}{n}\right)$.
    \end{enumerate}
\end{tcolorbox}

\begin{tcolorbox}[title=Exercice 4, title filled=false, colframe=darkgreen, colback=darkgreen!10!white]
    Soit $P \in \mathbb{R}[X]$ tel que pour tout $x \in \mathbb{R}, P(x) \geq 0$.
    \begin{enumerate}
        \item Montrer que si $P \neq 0$, alors toute racine réelle de $P$ est de multiplicité paire.
        \item En déduire que $P=A^2+B^2$, avec $(A, B) \in(\mathbb{R}[X])^2$.
    \end{enumerate}
\end{tcolorbox}


\end{document}