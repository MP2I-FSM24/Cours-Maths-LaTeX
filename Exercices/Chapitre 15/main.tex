\documentclass{report}

\usepackage{amsmath}
\usepackage{amssymb}
\usepackage{stmaryrd}
\usepackage{geometry}
\usepackage{tkz-tab}
\usepackage{tcolorbox}

\geometry{a4paper, left=20mm, right=20mm, top=20mm, bottom=20mm}

\begin{document}
\setcounter{chapter}{15}
\setcounter{section}{1}
\section{Exercice}
\begin{align*}
    0 \leq x \left\lfloor \frac{1}{x} \right\rfloor \leq x \times \frac{1}{x} \\
    \text{donc } 0 \leq x \left\lfloor \frac{1}{x} \right\rfloor \leq 1 \\
\end{align*}
\begin{enumerate}
    \item Quand $x \in ]-1, 0[ \cup ]0, 1[$, on remarque que $\lim\limits_x \left\lfloor \frac{1}{x} \right\rfloor \geq \lim\limits_x \frac{1}{x}$. \\
    Ainsi, par théorème d'opérations on a :
    \begin{align*}
        \lim_{x\to 0} x \times \frac{1}{x} \leq \lim_{x\to 0} x \times \left\lfloor \frac{1}{x} \right\rfloor \leq \lim_{x\to 0} x \times \frac{1}{x}
    \end{align*}
    Donc d'après le théorème d'encadrement : 
    \begin{align*}
        \boxed{\lim_{x\to 0} x \left\lfloor \frac{1}{x} \right\rfloor = 1}
    \end{align*}

    \item Quand $x > 1$, $\left\lfloor \frac{1}{x} \right\rfloor = 0$. \\
    Donc : 
    \begin{align*}
        \forall x > 1, x \left\lfloor \frac{1}{x} \right\rfloor = 0
    \end{align*}
    Donc : 
    \begin{align*}
        \boxed{\lim_{x\to +\infty} x \left\lfloor \frac{1}{x} \right\rfloor = 0} 
    \end{align*}
\end{enumerate}

\end{document}